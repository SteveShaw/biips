%% Ent�te du mod�le de rapport INRIA
%% ---------------------------------
\documentclass[a4paper]{report}
\usepackage[latin1]{inputenc} %
\usepackage[T1]{fontenc} %
\usepackage{RRA4indus}
\usepackage[colorlinks=true]{hyperref}
% \usepackage[frenchb]{babel} % optionnel

%% packages persos
%%----------------
\usepackage{lmodern} % polices latin vectorielles
%\usepackage{geometry}
%\geometry{ hmargin=2.5cm, vmargin=1.5cm}
\usepackage{amsmath}
\usepackage{amsfonts}
\usepackage{amssymb}
%\usepackage{here} % place la figure ICI \begin{figure}[H]. Obsolete !!
%\usepackage{algorithm,algorithmic}
%\floatname{algorithm}{Algorithme}
%\renewcommand{\listalgorithmname}{Liste des algorithmes} 
\usepackage{bm} % bold maths, pour les lettres grecques minuscules : $\bm{theta}$
\usepackage[nottoc, notlof, notlot]{tocbibind} % Entrer la bibliographie dans la table des mati�res, mais pas tableofcontents, listoffigures, listoftables
\usepackage{alltt} % sorte de verbatim qui interpr�te les commandes latex : \begin{alltt}
\usepackage[usenames,dvipsnames]{color}
\usepackage[bottom]{footmisc} % permet de placer les footnotes en bas de la page quel que soit le remplissage de la page
\graphicspath{{./logos/},{./figures/}}
\usepackage{listings} % code highlights

%% Commandes persos
%% ----------------
\newcommand{\biips}{%\bsifamily
\textsl{BiiPS}}
% \newcommand{\bugs}{%\bsifamily
% {BUGS}}
% \newcommand{\jags}{%\bsifamily
% {JAGS}}
% \newcommand{\smctc}{%\bsifamily
% {SMCTC}}
\newcommand{\biipslogo}{biips_logo_tests6d_shadow}
\newcommand{\biipslogomini}{biips_logo_tests6d}

% \newcommand{\aprio}{\textit{a priori}}
% \newcommand{\apost}{\textit{a posteriori}}
\newcommand{\ie}{\textit{i.e.}}
\newcommand{\eg}{\textit{e.g.}}
\newcommand{\noeud}{n\oe{}ud}


%%
%% date de publication du rapport
\RRdate{\today}
%%
%% Cas d'une version deux
%% \RRversion{2}
%% date de publication de la version 2
%% \RRdater{Novembre  2006}

%%
\RRauthor{% les auteurs
 % Premier auteur, avec une note
Adrien Todeschini, Fran�ois Caron%\thanks{Footnote for first author}%
  % note partag\'ee (optionnelle)
%  \thanks[sfn]{Shared foot note}%
 % \and entre chaque auteur s'il y en a plusieurs
%  \and
%Philippe Louarn\thanks{Footnote for second author}%
 % r\'ef\'erence \`a la note partag\'ee
%\thanksref{sfn}
}
%%
%% Ceci apparait sur chaque page paire.
%\authorhead{Grimm \& Louarn}
%%
\RRtitle{\includegraphics[height = 4cm]{\biipslogo}\\
%\vspace{11pt}
Projet \biips{} :\\
Rapport de conception}
%% English title
\RRetitle{\includegraphics[height = 4cm]{\biipslogo}\\
%\vspace{11pt}
\biips{} project:\\
Software design report}

% \RRtitle{
% Projet \biips{} :\\
% Rapport technique (brouillon)\\
% \vspace{1cm}
% \includegraphics[height = 4cm]{\biipslogo}}
% %% English title
% \RRetitle{
% \biips{} Project:\\
% Technical report (draft)\\
% \vspace{1cm}
% \includegraphics[height = 4cm]{\biipslogo}}
%%
\titlehead{\includegraphics[height = 11pt]{\biipslogomini} ~ \biips{} Project: Software design report}
%%
%\RRnote{This is a note}
%\RRnote{This is a second note}
%%
\RRresume{Ce rapport d�crit les aspects techniques du projet \biips tels que la conception du logiciel. Ce projet est port� par l'\href{http://alea.bordeaux.inria.fr/}{�quipe ALEA} de l'\href{http://www.inria.fr/bordeaux/}{INRIA Bordeaux - Sud-Ouest} et constitue une \textit{Action de d�veloppement Technologique} INRIA.}

\RRabstract{This report describes the technical aspects of the \biips{} project including software design. This project is carried out by the \href{http://alea.bordeaux.inria.fr/}{team ALEA} at \href{http://www.inria.fr/bordeaux/}{INRIA Bordeaux} and supported by INRIA through an \textit{Action de d�veloppement Technologique}.}

%%
\RRmotcle{Inf�rence bay�sienne, M�thodes de Monte Carlo s�quentielles, Filtrage particulaire, Mod�les graphiques, Calcul G�n�rique sur un Processeur Graphique}

\RRkeyword{Bayesian Inference, Sequential Monte Carlo, Particle Filters, Graphical Models, General Purpose Graphics Processing Units}
%%
\RRprojet{ALEA}  % cas d'un seul projet
%% \RRprojets{Apics et Op\'era}
%\RRequipe{Alea}
\RRdomaine{1} % cas du domaine numero 1
\RRtheme{Mod�les et m�thodes stochastiques}

\RCBordeaux % centre de recherche Bordeaux - Sud Ouest
\RRNo{2010-BiiPS-02}


%% Tikz
%%%%%%%%%%%%
\usepackage{tikz}
\usetikzlibrary{arrows}
\usetikzlibrary{fit}					% fitting shapes to coordinates
\usetikzlibrary{backgrounds}	% drawing the background after the foreground



%% Macro pour num�roter les figures en fonction des sections
%\makeatletter
%\renewcommand{\thefigure}{\ifnum \c@section>\z@ \thesection.\fi
% \@arabic\c@figure}
%\@addtoreset{figure}{section}
%\makeatother

%%
\begin{document}
%%
%% \makeRR   % cas d'un rapport de recherche
\makeRT % cas d'un rapport technique.

\tableofcontents
\clearpage


%%%%%%%%%%%%%%%%%%%%%%%%%%%%%%%%%%%%%%%%%%%%%%%%%%%%%%%%%%%%%%%%%%%%%%%%%%%%%%%%%%%%%%%%%%%%%%%%%%%%%%%%%%
%%
%% Nomenclature
%%
%%%%%%%%%%%%%%%%%%%%%%%%%%%%%%%%%%%%%%%%%%%%%%%%%%%%%%%%%%%%%%%%%%%%%%%%%%%%%%%%%%%%%%%%%%%%%%%%%%%%%%%%%%
\chapter*{Nomenclature}
\addcontentsline{toc}{chapter}{Nomenclature}

% \section*{Acronymes}
\section*{Acronyms}
\label{nomencl_acr}

\begin{tabular}{p{0.12\textwidth}p{0.8\textwidth}}
DAG & {Directed Acyclic Graph} \\%(Graphe acyclique orient�)\\
ESS & {Effective Sample Size} \\%(Taille d'�chantillon effective)\\
(GNU) GPL & {GNU General Public License} \\%(Licence publique g�n�rale GNU)\\
(GNU) LGPL & {GNU Lesser General Public License} \\%(Licence publique moins g�n�rale GNU)\\
GPGPU & {Global-purpose Processing on GPU} \\%(Calcul g�n�rique sur GPU)\\
GPU & {Graphic Processing Unit} \\%(Processeur graphique)\\
HMM & {Hidden Markov Model} \\%(Mod�le de Markov cach�)\\
IDE & {Integrated Development Environment} \\%(Environnement de d�veloppement int�gr�)\\
MCMC & {Markov Chain Monte Carlo} \\%(Monte Carlo par cha�ne de Markov)\\
OS & {Operating System} \\%(Syst�me d'exploitation)\\
RNG & {Random Number Generator} \\
SIMD & {Same Instruction, Multiple Data} \\%(M�me instruction, donn�es multiples)\\
SMC & {Sequential Monte Carlo} \\%(Monte Carlo s�quentiel)\\
\end{tabular}


% \section*{Logiciels, biblioth�ques}
\section*{Softwares, libraries}
\label{nomencl_soft}

\begin{tabular}{p{0.12\textwidth}p{0.8\textwidth}}
\biips & {Bayesian inference with interacting Particle Systems} \\%(Inf�rence bay�sienne avec des syst�mes de particules en interaction)\\
BGL & {Boost Graph Library}\\
BLAS & {Basic Linear Algebra Subprograms} \\%(Sous-programmes d'alg�bre lin�aire basiques)\\
BUGS & {Bayesian inference Using Gibbs Sampling} \\%(Inf�rence bay�sienne utilisant l'�chantillonnage de Gibbs)\\
CUDA & {Compute Unified Device Architecture} \\%(Architecture mat�rielle unifi�e pour le calcul)\\
CDT & {Eclipse C/C++ Development Tooling} \\%(Outillage de d�veloppement C/C++ pour Eclipse)\\
Flex & {Fast Lexical analyzer} \\%(Analyseur lexical rapide)\\
GCC & {GNU Compiler Collection} \\%(Collection de compilateurs GNU)\\
G++ & GNU C++ Compiler \\
GSL & {GNU Scientific Library} \\%(Biblioth�que scientifique GNU)\\
JAGS & {Just Another Gibbs Sampler} \\%(Juste un autre �chantillonneur de Gibbs)\\
JRE & Java Runtime Environment \\
QWT & {Qt Widgets for Technical applications} \\%(Composants d'interface graphique Qt pour applications techniques)\\
SMCTC & {SMC Template Class} \\%(Classe SMC g�n�rique)\\
SSH & Secure Shell \\
STL & {Standard Template Library} \\%(Biblioth�que standard g�n�rique)\\
SVN & Subversion \\
Yacc & {Yet another compiler compiler} %(Encore un autre compilateur de compilateur)
\end{tabular}



%%%%%%%%%%%%%%%%%%%%%%%%%%%%%%%%%%%%%%%%%%%%%%%%%%%%%%%%%%%%%%%%%%%%%%%%%%%%%%%%%%%%%%%%%%%%%%%%%%%%%%%%%%
%%
%% Introduction
%%
%%%%%%%%%%%%%%%%%%%%%%%%%%%%%%%%%%%%%%%%%%%%%%%%%%%%%%%%%%%%%%%%%%%%%%%%%%%%%%%%%%%%%%%%%%%%%%%%%%%%%%%%%%
\chapter*{Introduction}
\addcontentsline{toc}{chapter}{Introduction}

This report describes the technical aspects of the \biips{}\footnote{Bayesian inference with interacting Particle Systems} project including software design. It follows a previous functional specifications report \cite{biips_specifications_2010}, which only deals with \textbf{what} the software is expected to do, from the users point of view. In this report, we try to address the question: \textbf{how} it actually does it.

\paragraph{}
Chapter \ref{architecture} presents the development aspects including the software architecture, to understand what \biips{} software is composed of, and to understand the technical choices in terms of programming language and used libraries.

% \paragraph{}
% Chapter \ref{reference} is a short reference guide presenting the most important classes and design concepts of \biips Core library and the module \biips Base.

\paragraph{}
In chapter \ref{examples}, we present a simple example in order to understand how to write C++ code using \biips{} to instantiate a graphical model and run a SMC\footnote{Sequential Monte Carlo} algorithm.

\paragraph{}
Chapter \ref{developers} is a quick developers guide dealing with how to obtain the source code, compile and test \biips{} in a Linux environment. 

\paragraph{}
In chapter \ref{todo}, we sum up the remaining tasks, focusing on the BUGS language compiler.

%%%%%%%%%%%%%%%%%%%%%%%%%%%%%%%%%%%%%%%%%%%%%%%%%%%%%%%%%%%%%%%%%%%%%%%%%%%%%%%%%%%%%%%%%%%%%%%%%%%%%%%%%%
%%
%% Chapter : Architecture
%%
%%%%%%%%%%%%%%%%%%%%%%%%%%%%%%%%%%%%%%%%%%%%%%%%%%%%%%%%%%%%%%%%%%%%%%%%%%%%%%%%%%%%%%%%%%%%%%%%%%%%%%%%%%
\chapter{Architecture}
\label{architecture}


\section{Existing softwares}

\subsection{BUGS}
Quoting the OpenBUGS Project web page\footnote{\href{http://www.openbugs.info}{http://www.openbugs.info}}:

\begin{quote}
\textbf{Overview...}\\
BUGS\footnote{Bayesian inference Using Gibbs Sampling} is a software package for performing Bayesian inference Using Gibbs Sampling. The user specifies a statistical model, of (almost) arbitrary complexity, by simply stating the relationships between related variables. The software includes an 'expert system', which determines an appropriate MCMC (Markov chain Monte Carlo) scheme (based on the Gibbs sampler) for analysing the specified model. The user then controls the execution of the scheme and is free to choose from a wide range of output types.

\textbf{Versions...}\\
There are two main versions of BUGS, namely WinBUGS and OpenBUGS [...], an open-source version of the package, on which all future development work will be focused. OpenBUGS, therefore, represents the future of the BUGS project. [...]
Note that software exists to run OpenBUGS (and analyse its output) from within both R and SAS, amongst others. [...]

\textbf{How it works...}\\
The specified model belongs to a class known as Directed Acyclic Graphs (DAGs), for which there exists an elegant underlying mathematical theory. This allows us to break down the analysis of arbitrarily large and complex structures into a sequence of relatively simple computations. BUGS includes a range of algorithms that its expert system can assign to each such computational task.
[...]
\end{quote}

Its architecture is presented in \cite{lunn_winbugs_2000} but its code, written in Component Pascal, is not easily accessible and not well documented.


\subsection{JAGS}
JAGS\footnote{Just Another Gibbs Sampler}, written by Martyn Plummer, is a \textquotedblleft{}not wholly unlike BUGS\textquotedblright{} open source software (GPL\footnote{GNU General Public License} licensed). JAGS uses essentially the same model description language but it has been completely re-written. Its sources are in C++ and better documented.


\subsection*{}
\biips{} project is strongly inspired by both BUGS and JAGS. Indeed, we intend to propose SMC methods as an alternative to MCMC methods in bayesian graphical models inference, \ie{} we mainly want to change the core engine of the software. As a consequence, many parts of \biips{} are common with BUGS and JAGS, and even if they are re-written to suit with our goals, their design is very close. Also, some C++ sources of JAGS can eventually be re-used.


\section{Short functional specifications}

As a reminder, figure~\ref{fig:specgen} shows the input/output flow of the \biips{} software. See \cite{biips_specifications_2010} for more details.

\begin{figure}[h!]
\begin{center}
\tikzstyle{input}=[rectangle, rounded corners,
                                    thick,
                                    text width=2.5cm,
                                    draw=blue!80,
                                    fill=blue!20,
                                    text centered,
                                    font=\large,
                                    ]
\tikzstyle{processing}=[rectangle, rounded corners,
                                    thick,
                                    text width=2.5cm,
                                    draw=green!80,
                                    fill=green!20,
                                    text centered,
                                    font=\large,
                                    ]
\tikzstyle{output}=[rectangle, rounded corners,
                                                thick,
                                                text width=2.5cm,
                                                draw=orange!80,
                                                fill=orange!25,
                                                text centered,
                                    font=\large,]
\begin{tikzpicture}[node distance=4cm,auto,>=latex']
\node[input] (stat)   {Statistical\\Model};
\node[input] (data)   [below of= stat,node distance=1.2cm] {Data\\~};
\node (param)   [input,below of=data,node distance=1.2cm] {Parameters\\~};
\node (biips)   [processing,right of=data] {~\\\biips\\~};
\node (text) [below of=biips, node distance=1.2cm, text=green, font=\normalsize] {Black box};
\node (out)   [output,right of=biips] {Summary\\Statistics};
 \path[->] (stat) edge[thick] (biips);
 \path[->] (data) edge[thick] (biips);
 \path[->] (param) edge[thick] (biips);
 \path[->] (biips) edge[thick] (out);
 \end{tikzpicture}
 \caption{Input/output flow. Inputs are in blue and output in orange.}
 \label{fig:specgen}
 \end{center}
 \end{figure}


\paragraph{}
More precisely:

\paragraph{Input:}
\begin{itemize}
 \item Statistical model: it should be written in BUGS language. Inspired by S, this language is used by the wide community of BUGS software users. So we must consider the advantage for them to re-use, as they are, their model definitions written in this language. Moreover, JAGS is based on this language (except minor differences).
 \item Data: for the same reason stated above, we should input data from a textual S language type format.
 \item Parameters: a console interface should provide commands to specify the parameters of the algorithm as well as controls of the software.
\end{itemize}

\paragraph{Output:}
\begin{itemize}
 \item Summary Statistics: they should be output in textual S language type, table or graphical format.
\end{itemize}

\paragraph{}
\biips{} applies SMC methods in an \textquotedblleft{}automatic\textquotedblright{} manner, \ie{} roughly as a black box, with default choice of the tuning parameters of the algorithm.


\section{Software architecture}
The software architecture has been divided into bricks, as shown in figure \ref{fig:bricks}, each one bringing a new level of functionality. The development is done bottom-up.

\begin{figure}[h!]
\begin{center}
\tikzstyle{replace}=[rectangle, rounded corners,
                                    thick,
                                    text width=3cm,
                                    draw=blue!80,
                                    fill=blue!20,
                                    text centered,
                                    font=\large,
                                    ]
\tikzstyle{done}=[rectangle, rounded corners,
                                    thick,
                                    text width=3cm,
                                    draw=green!80,
                                    fill=green!20,
                                    text centered,
                                    font=\large,
                                    ]
\tikzstyle{todo}=[rectangle, rounded corners,
                                                thick,
                                                text width=3cm,
                                                draw=orange!80,
                                                fill=orange!25,
                                                text centered,
                                    font=\large,]

\tikzstyle{donetest}=[rectangle, rounded corners,
                                    thick,
                                    text width=3cm,
                                    draw=green!80,
                                    %fill=green!20,
                                    text=green!80,
                                    text centered,
                                    font=\large,
                                    ]

\tikzstyle{todotest}=[rectangle, rounded corners,
                                                thick,
                                                text width=3cm,
                                                draw=orange!80,
                                                %fill=orange!25,
                                                text=orange!80,
                                                text centered,
                                    font=\large,]
\begin{tikzpicture}[node distance=1.4cm,auto,>=latex']
\node (smctc) [replace]   {SMCTC\\ \small (GPL)};
\node (core)   [done,above of= smctc] {\biips\\Core};
\node (base)   [done,above of= core] {\biips\\Base};
\node (tests)   [donetest,right of=base,node distance=4cm] {Functional\\Tests 1};
\node (comp)   [todo,above of=base] {BUGS language\\Compiler};
\node (tests1)   [todotest,right of=comp,node distance=4cm] {Functional\\Tests 2};
\node (tihm)   [todo,above of=comp] {Text\\Interface};
\node (tests2)   [todotest,right of=tihm,node distance=4cm] {Functional\\Tests 3};
\node (gihm)   [todo,above of=tihm] {Graphical\\Interface};
\node (tests3)   [todotest,right of=gihm,node distance=4cm] {Functional\\Tests 4};
\node (repl) [left of=smctc,node distance=3cm, text=blue, font=\normalsize] {To replace};
\node (done) [left of=base,node distance=3cm, text=green, text centered, text width=3cm, font=\normalsize] {On going};
% \node (stat) [below of=done,node distance=.7cm, text=green, text centered, text width=3cm, font=\normalsize] {\small 73 fichiers\\$\approx$ 4500 lignes};
\node (todo) [left of=tihm,node distance=3cm, text=orange, font=\normalsize] {To do};
 \path[->] (core) edge[thick] (smctc);
 \path[->] (base) edge[thick] (core);
 \path[->] (tests) edge[thick] (base);
 \path[->] (comp) edge[thick] (base);
 \path[->] (tests1) edge[thick] (comp);
 \path[->] (tihm) edge[thick] (comp);
 \path[->] (tests2) edge[thick] (tihm);
 \path[->] (gihm) edge[thick] (tihm);
 \path[->] (tests3) edge[thick] (gihm);
\end{tikzpicture}
\label{fig:bricks}
\caption{\biips{}: software bricks}
\end{center}
\end{figure}
 
\begin{itemize}
 \item \textbf{\biips Core} is currently built over \textbf{SMCTC}\footnote{SMC Template Class} (GPL) in order to accelerate the development, but SMCTC has to be replaced afterwards. \biips Core library contains the core components of the program, \ie{} common types and classes used to define a bayesian graphical model and run a SMC sampler algorithm in order to estimate its posterior distribution.
 \item \textbf{\biips Base} is the first module which contains concrete classes corresponding to the abstract classes defined in the \biips Core library. \\
The complexity of the statistical models that can be defined with \biips{} increases with the number of concrete classes implemented in this module. In Addition, this architecture is intended to be extensible, \ie{} other modules (like \biips Base) with new features could be added afterwards.
 \item The \textbf{BUGS language Compiler} is essential for defining models without having to compile a C++ source code every time. See \ref{jags_compiler} for more details on the re-usability of JAGS compiler.
 \item The \textbf{Text interface} consists of a command line interpreter, which needs its own language. We can take ideas on JAGS terminal which is based on a Stata like syntax. This brick represents a common layer for all interface with other softwares such as Matlab or R. Moreover, batch treatments could be done thanks to this text interface, \ie{} submitting several commands at once in a script.
 \item On top of this text interface, a \textbf{Graphical interface} (\eg{} Qt based) would make the software standalone. It should implement a graphical point-and-click editor of bayesian graphical models.
\end{itemize}

\paragraph{}
Each time a new brick is built, some functional tests are performed according to the new features implemented.
\begin{itemize}
 \item \textbf{Functional Tests 1} based on \biips Base library implement different models specified in \cite{biips_specifications_2010} directly, by manual instantiations of the C++ classes. Chapter \ref{examples} details one of those tests. Section \ref{testing} explains a more rigorous procedure for testing \biips{} SMC algorithms on HMM models. The latter is implemented in \texttt{BiipsTest} program explained in section \ref{biipstest} of chapter \ref{developers}.
 \item \textbf{Functional Tests 2} of the Compiler will be based on BUGS language text files defining the same models as stated above. An ultimate objective would be to compile all the test examples of the BUGS software and compare the results.
 \item \textbf{Functional Tests 3} of the Text interface will use textual commands of the text interface to run examples.
 \item \textbf{Functional Tests} of the Graphical interface will consist in running examples using clicks on menus and buttons of the graphical interface.
\end{itemize}

\section{Testing procedure}
\label{testing}

We need a testing procedure saying, yes or no, the program is correct, in an automatic manner, \ie{} without having to look at the results. This procedure will be launched after each modification of the code and assures us that no bugs were introduced since the last stable version. SMC algorithms compute an approximation of the true filtering/smoothing posterior distribution. They are intrinsically erroneous and the errors are random. How can we measure the quality of the estimation? How can we say the program is right? The requirements obviously depend on the SMC algorithm, \textit{i.e} the number of particles, the re-sampling method and the type of exploration (mutation) used. They also depend on the model and the observed values. As a consequence, there is no universal criteria: each \textbf{configuration} \textit{i.e} \{model, set of observations, SMC algorithm\}, have its own criteria.

\paragraph{}
We propose that, for each \textbf{configuration}, we run $n_{SMC}$ i.i.d. SMC algorithms and compute the error versus the true posterior mean. For instance, let us consider a HMM where $x_{0:T}$ is the state. Let us call $Z$ the error of one SMC run:
$$
 Z = N \sum_{t=0}^{T} (\hat{x_t}-\bar{x_t})^T \varSigma_t^{-1} (\hat{x_t}-\bar{x_t})
$$
where $N$ is the number of particles,\\
$\hat{x_t} = \sum_{i=1}^N w^{(i)} x_t^{(i)}$ is the SMC posterior mean estimate,\\
$\bar{x_t}$, and $\varSigma_t$ are the true posterior mean and covariance.

\paragraph{}
In a linear Gaussian HMM, the latter are computed using Kalman equations. In non linear models, they are approximated using a fine grid method.

\paragraph{}
Suppose $Z_{1}, \ldots, Z_{n}$ are $n$ i.i.d. observations of the error, drawn from a correct SMC implementation (\textit{e.g.} under Matlab) of the given configuration. Let $F$ be the cumulative distribution function of this population.

\paragraph{}
Let now $Z'_{1}, \ldots, Z'_{n'}$ be $n'$ i.i.d. observations of the error drawn from \biips{} implementation of the given configuration. Let $F'$ be the cumulative distribution function of this population.

\paragraph{}
We want to test the \textquotedblleft{}goodness-of-fit\textquotedblright{} between $\{Z\}$ and $\{Z'\}$ samples, \ie{} the null hypothesis $H_0: F = F'$, versus $H_1: F \neq F'$. Using a Kolmogorov-Smirnov test, the statistic is:
$$
D_{n,n'} = \sup_{x} |F_n(x)-F'_{n'}(x)|
$$
where $F_n$ and $F'_{n'}$ are the empirical distribution functions of the first and second sample respectively.

\paragraph{}
The null hypothesis is rejected at level $\alpha$ if:
$$
\sqrt{\frac{n+n'}{nn'}} D_{n,n'} > K_{\alpha}
$$
where $K_{\alpha}$ is such that $\Pr(K \leq K_{\alpha})=1-\alpha$ and $K$ is distributed according to the Kolmogorov-Smirnov distribution.

\paragraph{}
Alternatively, we can accept $H_0$ if the p-value is greater than $\alpha$:
$$
\Pr(K \geq \sqrt{\frac{n+n'}{nn'}} D_{n,n'}) \geq \alpha
$$

\paragraph{}
Wish we test the correctness of \textbf{one} SMC run, we can check that its error does not exceed the empirical $1-\alpha$ quantile of the reference errors. Although this procedure is less powerful, it can be used for the sake of speed.

\section{Directories structure}

\textbf{\biips Core} code has been separated into the following folders:
\begin{itemize}
\item \textbf{common} contains the base types, classes and utilities common to all the other parts.
\item \textbf{distribution} contains the \texttt{Distribution} abstract class.
\item \textbf{function} contains the \texttt{Function} abstract class.
\item \textbf{model} contains the \texttt{Model} class.
\item \textbf{graph} contains the \texttt{Node} abstract class and its concrete classes as well as the \texttt{Graph} class.
\item \textbf{sampler} contains the \texttt{NodeSampler} abstract class (mutation applied to one node) and the \texttt{SMCSampler} class (derived from \texttt{smc::Sampler} from SMCTC) which applies the SMC algorithm to the whole graph.
\end{itemize}

\paragraph{}
\textbf{\biips Base} (as well as all future extension modules) code is divided into three folders :
\begin{itemize}
 \item \textbf{functions} contains concrete implementations of \texttt{Function} class, such as operators and usual mathematical functions. 
 \item \textbf{distributions} contains concrete implementations of\texttt{Distribution} class, such as usual univariate and multivariate distributions.
 \item \textbf{samplers} contains concrete implementations of \texttt{NodeSampler} class which update stochastic nodes in the graph.
\end{itemize}



\section{Development environment}

\begin{itemize}
 \item Language: We chose C++ for the object-oriented paradigm essential for \biips{} design, its flexibility, portability, popularity and reasonably fast execution.
 \item OS\footnote{Operating System}: Linux Ubuntu 32 bits, yielding a complete development environment, including GCC\footnote{GNU Compiler Collection} and make.
 \item IDE\footnote{Integrated Development Environment}: Eclipse CDT\footnote{Eclipse C/C++ Development Tooling}.
 \item The software will be ported to other platforms via a cross-platform tool such as CMake.
 \item Hosted on GForge INRIA, using SVN\footnote{Subversion} as version control system.
\end{itemize}



\section{Used libraries}

Here are the libraries I use in the development, according to the following criteria: they must be portable and non restrictive for the license as we don't know which license will be used for \biips{} beforehand.

\begin{itemize}
 \item The \textbf{Standard C++ library} notably the \textbf{STL}\footnote{Standard Template Library} and its containers.
 \item \textbf{Boost} is a collection of template libraries developed by an active community of the greatest C++ developers. It is a well recognized reference influencing the C++ standards.\\\\
\begin{minipage}{0.8\textwidth}
\begin{quote}
\textquotedblleft...one of the most highly regarded and expertly designed C++ library projects in the world.\textquotedblright{} 
\end{quote}
\begin{flushright}\textminus{} Herb Sutter and Andrei Alexandrescu, C++ Coding Standards.
\end{flushright}
\end{minipage}\\\\
Moreover, its license is not restrictive and most of the libraries are headers-only. Here are some used components :
  \begin{itemize}
  \item \textbf{Shared\_ptr} is an implementation of a shared pointer. In our case, the sequential particle re-samplings result in lot of data being common to many particles. Thanks to the shared ownership, this data is not duplicated and we do not need to manage their destruction.
  \item \textbf{uBlas} provides vector and matrices containers as well as linear algebra operators conforming to BLAS\footnote{Basic Linear Algebra Subprograms}.
  \item \textbf{BGL}\footnote{Boost Graph Library} provides graph structures and algorithms such as the topological order or the cycle detection.
  \item \textbf{Random} allows pseudo random-numbers generation according to several distributions.
  \item \textbf{Math toolkit} provides advanced mathematical functions as well as a collection of probability density functions.
  \item \textbf{Accumulators} is a framework allowing to accumulate data in order to compute sums, means, variances, etc.
  \item \textbf{Operators} facilitates the implementation of numerical operators into a class.
  \item \textbf{Bimap} is a bidirectional maps library.
  \item \textbf{Numeric Conversion} is a collection of tools to describe and perform conversions between values of different numeric types.
  \item \textbf{Program\_options} allows program developers to obtain program options, that is (name, value) pairs from the user, via conventional methods such as command line and configuration file.
  \item \textbf{Test} provides a matched set of components for writing test programs, organizing tests in to simple test cases and test suites, and controlling their runtime execution.
  \item Most of these libraries have extensions in \textbf{Boost sandbox}. Those extensions non distributed in the official release (and not submitted to review) are free as well.
  \end{itemize}
 \item \textbf{QWT}\footnote{Qt Widgets for Technical applications} is based on \textbf{Qt} and allows us to plot scientific graphics.
 \item \textbf{SMCTC} is used at first to accelerate the development but will be replaced afterwards by our own code. It uses \textbf{GSL}\footnote{GNU Scientific Library} for pseudo random numbers generation.
\end{itemize}


% \section{Design patterns}
% 
% Some used design patterns:
% \begin{itemize}
%  \item Factory permet de g�n�raliser l'instanciation d'un type de base dans le but d'avoir un code extensible. Il est utilis� pour les mutations : il n'existe qu'une instance de factory pour les mutations et chaque mutation X (d�riv�e de la class de base) doit installer sa propre instance de factory dans l'instance g�n�rale pour que cette derni�re cr�e une mutation X. Ainsi le code responsable de cr�er les mutations n'a pas besoin de conna�tre le constructeur de chaque mutation d�riv�e.
%  \item Singleton est utilis� lorqu'une classe ne doit poss�der qu'une seule instance. Celle-ci met son constructeur en membre priv� ou prot�g� et dispose d'une m�thode Instance qui retourne la seule instance de la classe. Ce patron est utilis� pour les Factorys mais �galement pour les Fonctions ou Distributions.
%  \item Visitor est un patron qui permet de faire du polymorphisme (le code ex�cut� d�pend du type d'objet). Il est utilis� pour les Nodes. Ceux-ci sont stock�s dans un tableau de Nodes (classe de base) mais ils peuvent �tre de classe d�riv�e ConstantNode, LogicalNode ou StochasticNode. Pour faire une action diff�rente sur chaque type, on d�rive la classe NodeVisitor et on red�finit un m�thode Visit diff�rente pour les trois types de Node. Ce proc�d� extensible permet de ne pas avoir � rajouter de fonctions membres virtuelles et d'�viter les \textit{dynamic\_cast}.
% \end{itemize}
% 




%%%%%%%%%%%%%%%%%%%%%%%%%%%%%%%%%%%%%%%%%%%%%%%%%%%%%%%%%%%%%%%%%%%%%%%%%%%%%%%%%%%%%%%%%%%%%%%%%%%%%%%%%%
%%
%% Chapter : Reference Manual
%%
%%%%%%%%%%%%%%%%%%%%%%%%%%%%%%%%%%%%%%%%%%%%%%%%%%%%%%%%%%%%%%%%%%%%%%%%%%%%%%%%%%%%%%%%%%%%%%%%%%%%%%%%%%
% \chapter{Reference manual}
\label{reference}

\section{\biips Core library}

\subsection{Common types and classes}

\subsubsection{Basic types}
Declared in header \texttt{"common/Types.hpp"}

\subsubsection{Data handling}
Declared in header \texttt{"common/MultiArray.hpp"}

\subsection{Function}
Declared in header \texttt{"function/Function.hpp"}

\subsection{Distribution}
Declared in header \texttt{"distribution/Distribution.hpp"}

\subsection{Nodes}
Declared in header \texttt{"graph/Node.hpp"}

\subsection{Graph}
Declared in header \texttt{"graph/Graph.hpp"}

\subsection{Node Visitors}
Declared in header \texttt{"graph/NodeVisitor.hpp"}

\subsection{Node Samplers}
Declared in header \texttt{"sampler/NodeSampler.hpp"}

\subsection{SMC Sampler}
Declared in header \texttt{"sampler/SMCSampler.hpp"}

\subsection{Accumulators}
Declared in header \texttt{"sampler/Accumulator.hpp"}


\section{\biips Base module}

\subsection{Functions}

\subsection{Distributions}

\subsection{Samplers}



%%%%%%%%%%%%%%%%%%%%%%%%%%%%%%%%%%%%%%%%%%%%%%%%%%%%%%%%%%%%%%%%%%%%%%%%%%%%%%%%%%%%%%%%%%%%%%%%%%%%%%%%%%
%%
%% Chapter : Applications Examples
%%
%%%%%%%%%%%%%%%%%%%%%%%%%%%%%%%%%%%%%%%%%%%%%%%%%%%%%%%%%%%%%%%%%%%%%%%%%%%%%%%%%%%%%%%%%%%%%%%%%%%%%%%%%%
\chapter{Tutorial example}
\label{examples}

\lstset{language = C++, %
 basicstyle=\small\ttfamily, %
keywordstyle=\bfseries\color{RedViolet}, %
identifierstyle=,%
 commentstyle=\color{ForestGreen},%
stringstyle=\color{blue},%
%column=fixed,%
basewidth={0.5em,0.45em},%
%numbers=left, numberfirstline=false, numberstyle=\tiny, stepnumber=10, numbersep=5pt%
}

In this chapter, we present C++ code using the \biips Core library and the first module of extensions, \biips Base. This should quickly introduce you to defining a statistical model and run the SMC algorithm. This implies a C++ compilation for each defined model. Also, interactions may be poor. However, these examples will serve as a tutorial for understanding the the classes and their use. Future versions of \biips{} will use a BUGS language compiler allowing a more user-friendly model definition, without any C++ compilation.

\section{Preamble}

The source files will have to \verb=#include= the headers of \biips Core and the modules used, \eg{} \biips Base.
\begin{lstlisting}
#include "BiipsCore.hpp"
#include "BiipsBase.hpp"
\end{lstlisting}

\paragraph{}
As all \biips{} names reside in the \verb=Biips= namespace, a \verb=using= command will allow us to write \biips{} names without the prefix \verb=Biips=
\begin{lstlisting}
using namespace Biips;
\end{lstlisting}


\section{A simple particle filtering example}

Let's consider the following linear gaussian state space model 1D (\ie{} the first functional test defined in \cite{biips_specifications_2010}) :

$$
X_0\sim \mathcal{N}(0,1)
$$
For $t=1,\ldots,20$
$$
X_t|X_{t-1}\sim \mathcal{N}(X_{t-1},1)
$$
$$
Y_t|X_{t}\sim \mathcal{N}(X_{t},0.5)
$$


\paragraph{}
The C++ code of this example is in appendix \ref{miniex}. In the following, we detail this code step-by-step. It doesn't have any input/output instructions in order to focus on understanding the basic rules of using \biips{} library.


\subsection{Instantiate the model with \biips}

\begin{enumerate}
 \item Define the data :
\begin{lstlisting}
// Final time :
Size t_max = 20;
// Initial mean of X[0]
Scalar mu_x0_val = 0;
// Variance of X[0] :
Scalar sig_x0_val = 1;
// Variance of X[t] | X[t-1] :
Scalar sig_x_val = 1;
// Variance of Y[t] | Y[t] :
Scalar sig_y_val = 0.5;
\end{lstlisting}
\texttt{Size} and \texttt{Scalar} are \verb=typedefs= of integer and floating-point types.


 \item Declare a \texttt{FunctionTable} and a \texttt{DistributionTable} objects. They will contain Function and Distribution instances and allow us to access them by their name in BUGS language.
\begin{lstlisting}
FunctionTable func_tab;
DistributionTable dist_tab;
\end{lstlisting}


 \item Load the Base module. This will add the distributions and functions declared in \biips Base library to the \texttt{FunctionTable} and \texttt{DistributionTable} objects. As a consequence, \texttt{Graph} will be able to instantiate logical and stochastic nodes based on those functions and distributions. This will also push \texttt{NodeSamplerFactory} instances in the \texttt{SMCSampler::NodeSamplerFactories} static member. The latter holds factories of sampling methods for one node of the graph in a priority order.
\begin{lstlisting}
loadBaseModule(func_tab, dist_tab);
\end{lstlisting}


 \item Declare the \texttt{Graph} object.
\begin{lstlisting}
Graph graph;
\end{lstlisting}


 \item Add the constant nodes to the graph.
\begin{lstlisting}
NodeId mu_x0 = graph.AddConstantNode(MultiArray(mu_x0_val));
NodeId sig_x0 = graph.AddConstantNode(MultiArray(sig_x0_val));
NodeId sig_x = graph.AddConstantNode(MultiArray(sig_x_val));
NodeId sig_y = graph.AddConstantNode(MultiArray(sig_y_val));
\end{lstlisting}
The \texttt{AddConstantNode} method of \texttt{Graph} class must take a \texttt{MultiArray} object as argument. This object represents a n-dimensional data that aggregates (with shared pointers) its dimension (a \texttt{DimArray} object, which is an array of positive integers) and its values (a \texttt{ValArray} object, which is a \verb!std::vector! based container of scalar values, fitted with element-wise numerical operators and functions). It can be directly constructed from a \texttt{Scalar}. The unique identifier of the node is stored in a \texttt{NodeId} object, which is a \verb=typedef= of integer type.


\item Create stochastic \texttt{NodeId} collections to store $X$ and $Y$'s components identifiers.
\begin{lstlisting}
Types<NodeId>::Array x(t_max+1);
Types<NodeId>::Array y(t_max);
\end{lstlisting}
The template \verb=Types<class T>= structure defines derived types from its template parameter type such as \texttt{Array} (a \verb=typedef= for \verb=std::vector<T>=), \texttt{Ptr} (a typedef for \verb=boost::shared_ptr<T>=), etc.

\item Create a \texttt{NodeId} array to handle the parents of each stochastic node.
\begin{lstlisting}
Types<NodeId>::Array params(2);
\end{lstlisting}
All stochastic nodes will have 2 parents.


\item Add $X_0$ stochastic node to the graph.
\begin{lstlisting}
params[0] = mu_x0;
params[1] = sig_x0;
x[0] = graph.AddStochasticNode(P_SCALAR_DIM, dist_tab["dnormvar"], params, false);
\end{lstlisting}                                                                                                                                                                                                                                                                                                                                                                                                                                                                                                                                                                                                                                                                                                                                                                      
The \texttt{AddStochasticNode} method of \texttt{Graph} class takes as arguments: the dimension of the node, a string identifying the distribution in the \texttt{DistributionTable} object (\eg{} \verb="dnormvar"= for the normal distribution\footnote{Conforming to the BUGS language, \texttt{"dnorm"} keyword defines a normal distribution which parameters are mean and precision (inverse of variance). We added the \texttt{"dnormvar"} keyword that takes the variance as second parameter.}), the list of its ordered parameters identifiers (mean and variance/precision for the normal distribution) and a boolean set to true if the node is observed. Alternatively, we can pass a \texttt{MultiArray} object as last argument defining the value of the observation.\\
The dimension is stored in a \texttt{DimArray} object. \verb=P_SCALAR_DIM= is a constant shared pointer exposed by \biips Core library. It holds a \texttt{DimArray} object for scalars. This \texttt{DimArray} object is dynamically allocated and handled by a shared pointer so its ownership will be shared by all the nodes. It is declared and allocated as
\begin{lstlisting}
const DimArray::Ptr P_SCALAR_DIM(new DimArray(1,1));
\end{lstlisting}
The first argument is the number of dimensions (size of the array) and the second is the length of the dimensions (value of the elements of the array), \ie{} \verb=P_SCALAR_DIM= is here defined as a one element array containing 1.

\item Add the other stochastic nodes to the graph.
\begin{lstlisting}
for (Size t=1; t<t_max+1; ++t)
{
  // Add X[t]
  params[0] = x[t-1];
  params[1] = sig_x;
  x[t] = graph.AddStochasticNode(P_SCALAR_DIM, dist_tab["dnormvar"], params, false);

  // Add Y[t]
  params[0] = x[t];
  params[1] = sig_y;
  y[t-1] = graph.AddStochasticNode(P_SCALAR_DIM, dist_tab["dnormvar"], params, true);
}
\end{lstlisting}


\item Check if the graph has cycles.
\begin{lstlisting}
Bool has_cycle = graph.HasCycle();
\end{lstlisting}

\item Build the graph.
\begin{lstlisting}
graph.Build();
\end{lstlisting}
After checking that the graph has no cycle, this will build the stochastic edges of the internal Boost graph structure allowing us to access stochastic parent/children relationships. Then the nodes are sorted in topological order where the parents before their children in a sequence.


\item Generate data.
\begin{lstlisting}
// Declare a random number generator, initialized with an integer seed
Rng my_rng(0);

// Sample values according to the stochastic nodes prior distribution
NodeValues gen_values = graph.SampleValues(&my_rng);

// Set the observations values according to the generated values
graph.SetObsValues(gen_values);
\end{lstlisting}
Here we generate a random sample for the whole graph. The values of each node are stored in a \texttt{NodeValues} object. This object is nothing but an array storing \texttt{ValArray::Ptr} objects (\ie{} shared pointers of \texttt{ValArray} objects) where the indices of the array correspond to the node identifiers (\ie{} \verb=gen_values[3]= is a shared pointer to the values of node 3). The call to \texttt{SetObsValues} method assigns their corresponding value in the \texttt{NodeValues} object to the observed stochastic nodes. We can easily produce a toy example this way. Changing the \texttt{Rng}\footnote{Random Number Generator} seed will change the sampled values.

\end{enumerate}




\subsection{Run the SMC algorithm}

\begin{enumerate}
 \item Define the number of particles.
\begin{lstlisting}
Size nb_particles = 1000;
\end{lstlisting}


 \item Declare the \texttt{SMCSampler} object.
\begin{lstlisting}
SMCSampler sampler(nb_particles, &graph, &my_rng);
\end{lstlisting}


 \item Initialize the \texttt{SMCSampler} object.
\begin{lstlisting}
sampler.Initialize();
\end{lstlisting}
In particular, this will build the \texttt{Node} sequence of the \texttt{SMCSampler} object. It consists of the unobserved stochastic nodes of the graph in a topological order (\ie{} all the parents of each node are placed before it in the sequence). Then, this will assign a sampling method to each node by trying, in the order of priority, the \texttt{Create()} method of each \texttt{NodeSamplerFactory} instance. If no advanced method can sample a node, the default prior \texttt{NodeSampler} implementation is assigned to it. The latter will sample the particles according to the prior distribution of the node.

 \item Declare and configure a \texttt{ScalarAccumulator} object.
\begin{lstlisting}
ScalarAccumulator stats_acc;
stats_acc.AddFeature(MEAN);
stats_acc.AddFeature(VARIANCE);
\end{lstlisting}
This object will be used to compute the summary statistics of the posterior distribution, based on the particle values and weights. We have added \texttt{MEAN} and \texttt{VARIANCE} features but many others can be added. Computations are made efficient by taking into account the dependent computations between several features.


 \item Declare \texttt{Scalar} arrays to store the posterior mean and variance estimates.
\begin{lstlisting}
Types<Scalar>::Array x_est_PF(t_max+1);
Types<Scalar>::Array x_var_PF(t_max+1);
\end{lstlisting}


 \item Iterate the SMC algorithm and extract summary statistics.
\begin{lstlisting}
for (Size t=0; t<t_max+1; ++t)
{
  // Iterate the SMC algorithm : resample (if needed) and sample one node of the sequence
  sampler.Iterate();

  // Accumulate particles corresponding to the last updated node
  sampler.Accumulate(x[t], stats_acc);

  // extract summary statistics of the filtering density
  x_est_PF[t] = stats_acc.Mean();
  x_var_PF[t] = stats_acc.Variance();
}
\end{lstlisting}
According to the model, the \texttt{SMCSampler}'s sequence is $X_0, X_1, \ldots, X_{20}$. Each time the \texttt{Iterate()} method is called, it re-samples (if needed) and samples the particles according to the \texttt{NodeSampler} method assigned to the current node. It then computes the new ESS\footnote{Effective Sample Size}. At step $t$, accumulation of node $k$ yields summary statistics of the posterior density $p(X_k|X_{0:t})$. Hence, accumulation of node $t$ at step $t$ corresponds to estimating the marginal filtering density $p(X_t|X{0:t})$.


\end{enumerate}


%%%%%%%%%%%%%%%%%%%%%%%%%%%%%%%%%%%%%%%%%%%%%%%%%%%%%%%%%%%%%%%%%%%%%%%%%%%%%%%%%%%%%%%%%%%%%%%%%%%%%%%%%%
%%
%% Chapter : Developers Manual
%%
%%%%%%%%%%%%%%%%%%%%%%%%%%%%%%%%%%%%%%%%%%%%%%%%%%%%%%%%%%%%%%%%%%%%%%%%%%%%%%%%%%%%%%%%%%%%%%%%%%%%%%%%%%
\chapter{Developers guide}
\label{developers}

\section{Required configuration}
\biips{} is currently developed and has only been tested under Ubuntu 10.04 (Lucid lynx) 32 bits. The following instructions will only concern this configuration.

\paragraph{}
First, activate optional software repositories:
\begin{quote}
Click the Ubuntu menu \emph{System $\rightarrow$ Administration $\rightarrow$ Software sources}. \\
Select the \emph{universe} and, \emph{multiverse} repositories.
\end{quote}

\subsection{Libraries}
Then, install the following libraries:

\begin{list}{}{}

\item[\textbf{GSL:}] type the following command in a terminal
\begin{verbatim}
sudo apt-get install libgsl0-dev
\end{verbatim}


 \item[\textbf{Boost:}] type the following command in a terminal
\begin{verbatim}
sudo apt-get install libboost-all-dev
\end{verbatim}


\item[\textbf{Qt:}] type the following command in a terminal
\begin{verbatim}
sudo apt-get install libqtcore4
sudo apt-get install libqtgui4
\end{verbatim}


\item[\textbf{QWT:}] type the following command in a terminal
\begin{verbatim}
sudo apt-get install libqwt5-qt4-dev
\end{verbatim}

\end{list}


\subsection{Development tools}
Finally, install the following development tools:

\begin{list}{}{}

\item[\textbf{JRE\footnote{Java Runtime Environment}:}] type the following command in a terminal
\begin{verbatim}
sudo apt-get install openjdk-6-jre
\end{verbatim}


\item [\textbf{Eclipse:}] type the following command in a terminal
\begin{verbatim}
sudo apt-get install eclipse
\end{verbatim}
The first launch of Eclipse will ask you for a workspace directory. You can accept the default one: \verb=~/workspace=.


\item [\textbf{G++\footnote{GNU C++ Compiler}:}] type the following command in a terminal
\begin{verbatim}
sudo apt-get install g++
\end{verbatim}


\item [\textbf{CDT plugin:}] in Eclipse
\begin{quote}
Click the top menu \emph{Help $\rightarrow$ Install new software}. \\
In the \emph{Work with} field, type: \verb=http://download.eclipse.org/tools/cdt/releases/galileo= \\
Click \emph{Add} \\
Select all the features (CDT Main Features, CDT Optional Features). \\
Click \emph{Next}, \emph{Next}, \emph{Accept}, \emph{Finish}. \\
Answer \emph{OK} to the \emph{Security warning} message. \\
Restart Eclipse.
\end{quote}
To open the C/C++ perspective
\begin{quote}
Click the top center (curved arrow shaped) button on the welcome page to go to the workbench. \\
Then, click the top menu \emph{Window $\rightarrow$ Open perspective $\rightarrow$ Other}. \\
Select C/C++.
\end{quote}
You should also set the following option preference
\begin{quote}
Click the top menu\emph{ Window $\rightarrow$ Preferences $\rightarrow$ Run/Debug $\rightarrow$ Launching}. \\
Unselect \emph{build (if required) before launching}.
\end{quote}
This will avoid long waiting times before launching an executable.


\item [\textbf{SSH\footnote{Secure Shell}:}] type the following command in a terminal
\begin{verbatim}
sudo apt-get install ssh
\end{verbatim}


\item [\textbf{SVN\footnote{Subversion}:}] type the following command in a terminal
\begin{verbatim}
sudo apt-get install subversion
sudo apt-get install subversion-tools
\end{verbatim}


\end{list}

\section{Obtaining the source code}
You must have registered your SSH public key in your GForge account.

\paragraph{}
Quoting the GForge FAQ\footnote{\href{http://siteadmin.gforge.inria.fr/FAQ.html\#Q6}{http://siteadmin.gforge.inria.fr/FAQ.html\#Q6}}:
\begin{quote}
\begin{itemize}
\item Generate a pair of rsa public/private keys with the ssh-keygen command: \verb=ssh-keygen -t rsa= (make sure you enter a lengthy passphrase, ie non null, when asked to).\\
\item Copy your public key in an easy-to access file: \verb=cp ~/.ssh/id_rsa.pub ~=.\\
\item Paste your public key in the gforge website. To do this, you need to go to your account and then go to the Account maintainance tab. At the bottom of the Account maintainance tab, you should see a Shell Account Information section which contains an [Edit keys] link. Paste your public key(s) in the empty field below and click the Update button.\\
\end{itemize}
Please, be aware that uploading your ssh public key on the server will not allow you to connect to the server immediately through ssh. To do so, you will need to wait at most 24h. If your connection is impossible 24h later, please, contact the server administrators.
\end{quote}

You can now make a SVN checkout of the whole \biips{} repository into your workspace, typing the following commands in a terminal:
\begin{quote}
\verb=cd ~/workspace= \\
\texttt{svn checkout svn+ssh://\textit{<Gforge login>}@scm.gforge.inria.fr/svn/biips}
\end{quote}


\paragraph{}
Then, import the Eclipse trunk projects with the following instructions:
\begin{quote}
Click the top menu \emph{File $\rightarrow$ Import}. \\
Select \emph{General $\rightarrow$ Existing Projects into Workspace}. \\
Click \emph{Next}. \\
In the \emph{Select root directory} field, select \verb=~/workspace/biips/trunk=.
\end{quote}
Four projects must appear and be selected (BiipsBase, BiipsCore, BiipsTest, smctc)
\begin{quote}
Click \emph{Finish}.
\end{quote}


\section{Compiling}
For the first compilation, select the Debug or Release configurations for the three BiipsCore, BiipsBase, BiipsTest projects:
\begin{quote}
Select the projects in the \emph{Project Explorer} view.\\
Click the top menu \emph{Project $\rightarrow$ Build Configurations $\rightarrow$ Set Active $\rightarrow$ Debug/Release}.
\end{quote}
Then build libraries and executables:
\begin{quote}
Click the top menu \emph{Project $\rightarrow$ Build all}.
\end{quote}

\paragraph{}
For later quicker builds:
\begin{quote}
Select your project in the \emph{Project Explorer} view.\\
Click the top menu \emph{Project $\rightarrow$ Make Targets $\rightarrow$ all $\rightarrow$ Build}.
\end{quote}
This will skip the checks of dependencies with other projects.

\section{Testing}
\label{biipstest}

\texttt{BiipsTest} implements the testing procedure presented in section \ref{testing} of chapter \ref{architecture}. It uses \textbf{Boost.Test} test driver and it can take several command-line arguments.
\begin{itemize}
 \item \verb=BiipsTest --help= displays help concerning the \textbf{Boost.Test} test driver.
 \item \verb=BiipsTest --help-test= or \verb=BiipsTest -h= displays help concerning \biips{} tests.
\end{itemize}

\paragraph{}
The latter produces the following message:
\begin{verbatim}
Usage: BiipsTest [<option>]... --model-id=<id> [<option>]...
  runs model <id> with default model parameters, with multiple <option> parameters.
       BiipsTest [<option>]... <config_file> [<option>]...
  runs test configuration from file <config_file>, with multiple <option> parameters.

Options, denoted here by <option>, are parsed from the general syntax:     --<long_key>[=<value>]
  or -<short_key>[=<value>]
where the bracketed text [...] is optional
and the angle bracketed text <...> must be replaced by the corresponding value.

Allowed options:

Generic options:
  -h [ --help-test ]           produces help message.
  --version                    prints version string.
  -v [ --verbose ] arg (=1)    verbosity level.
                               values:
                                0: none.
                                1: minimal.
                                2: high.
  --interactive                asks questions to the user.
                               applies when verbose>0.
  -s [ --show-plots ] arg (=0) shows plots, interrupting execution.
                               applies when n-smc=1.
                               values:
                                0: no plots are shown.
                                1: final results plots are shown.
                                2: all plots are shown.
  --step arg (=3)              execution step to be reached (if possible).
                               values:
                                0: samples or reads values of the graph.
                                1: computes or reads benchmark values.
                                2: runs SMC sampler, computes estimates and 
                                   errors vs benchmarks.
                                3: checks that errors vs benchmarks are 
                                   distributed according to reference SMC errors, 
                                   when n-smc>1. checks that error is lesser than a
                                   1-(reject-level) quantile of the reference SMC 
                                   errors, when n-smc=1.
  --check-mode arg (=filter)   errors to be checked.
                               values:
                                filter: checks filtering errors only.
                                smooth: checks smoothing errors only.
                                all: checks both.

Configuration:
  --model-id arg              model identifier.
                              values:
                               1: HMM 1D linear gaussian.
                               2: HMM 1D non linear gaussian.
                               3: HMM multivariate Normal linear.
                               4: HMM multivariate Normal linear 4D.
  --data-rng-seed arg         data sampler rng seed. default=time().
  --resampling arg (=2)       resampling method.
                              values:
                               0: MULTINOMIAL.
                               1: RESIDUAL.
                               2: STRATIFIED.
                               3: SYSTEMATIC.
  --ess-threshold arg (=0.5)  ESS resampling threshold.
  --particles arg (=1000 )    numbers of particles.
  --mutations arg (=optimal ) type of exploration used in the mutation step of 
                              the SMC algorithm.
                              values:
                               optimal: use optimal mutation, when possible.
                               prior: use prior mutation.
                               <other>: possible model-specific mutation.
  --n-smc arg (=1)            number of independent SMC executions for each 
                              mutation and number of particles.
  --smc-rng-seed arg          SMC sampler rng seed. default=time().
                              applies when n-smc=1.
  --prec-param                uses precision parameter instead of variance for 
                              normal distributions.
  --reject-level arg (=0.01)  accepted level of rejection in checks.
  --plot-file arg             plots pdf file name.
                              applies when n-smc=1.
\end{verbatim}

\paragraph{}
\textbf{Generic options} can only been taken from command-line and \textbf{Configuration} ones can be taken both from command-line and from a configuration file. The general syntax is: \verb!--<option>! or \verb!--<option>=<value>!, where the text in angle brackets must be replaced with appropriate content.

\paragraph{}
\textbf{Unregistered options}\footnote{Options whose name can not be known until reading it.} can also be read from the configuration file:
\begin{itemize}

\item Model parameters:
\begin{verbatim}
[model]
t.max = 20
\end{verbatim}
Matrix parameters are parsed from the following syntax
\begin{verbatim}
[model]
A = value11 value21; value12 value22
\end{verbatim}
where \verb!;! is a column separator.

\item Dimensions:
\begin{verbatim}
[dim]
x = 1
y = 2 2
\end{verbatim}
\verb!x! is scalar and \verb!y! is a $2 \times 2$ matrix

\item Data, \ie{} observed (and optionally unobserved) values of the model:
\begin{verbatim}
[data]
x = 1.01144 0.913546 -0.228905 0.493078 0.195882 -1.78846 0.611299 [...]
\end{verbatim}
Multi-dimensional values can be parsed from syntax:
\begin{verbatim}
[data]
y.1 = 1.01144 0.913546 -0.228905 0.493078 0.195882 -1.78846 0.611299 [...]
y.2 = 0 0.809148 0.886073 0.0694691 0.379583 0.245104 -1.24357 [...]
y.3 = 1 0.4 0.368421 0.366197 0.366038 0.366026 0.366025 [...]
y.4 = 0.386182 0.772364 0.680402 0.122153 0.26602 -0.044229 -0.8347 [...]
\end{verbatim}
where the dimensions are in row major order:
 \begin{itemize}
   \item \verb!y.1! line contains $y(1,1)$ values
   \item \verb!y.2! line contains $y(2,1)$ values
   \item \verb!y.3! line contains $y(1,2)$ values
   \item \verb!y.4! line contains $y(2,2)$ values
 \end{itemize}

\item Benchmark values, \ie{} the true posterior mean and variance values:
\begin{verbatim}
[bench.filter]
x = 0 0.809148 0.886073 0.0694691 0.379583 0.245104 -1.24357 [...]
var.x = 1 0.4 0.368421 0.366197 0.366038 0.366026 0.366025 [...]
[bench.smooth]
x = 0.386182 0.772364 0.680402 0.122153 0.26602 -0.044229 -0.8347 [...]
var.x = 0.57735 0.309401 0.290163 0.288782 0.288683 0.288676 0.288675 [...]
\end{verbatim}

\item Reference error samples, \ie{} obtained from a correct (under Matlab) SMC algorithm:
\begin{verbatim}
[errors.filter]
prior.500 = 29.8718 39.0415 26.4701 35.337 23.9243 22.4974 43.9526 [...]
optimal.500 = 40.0482 51.5672 26.4215 64.4748 21.4929 33.9718 38.2285 [...]
prior.1000 = 34.5009 16.0078 42.007 23.0936 33.778 13.8802 22.679 [...]
optimal.1000 = 19.6426 27.7204 28.2136 25.245 24.3569 34.7842 33.0246 [...]
[errors.smooth]
prior.500 = 39.9341 41.0893 24.1553 29.9975 21.479 27.6641 50.4908 [...]
optimal.500 = 31.9834 44.6602 26.0642 68.4008 28.5624 27.6433 45.0351 [...]
prior.1000 = 38.8716 14.5631 44.6106 23.2586 29.2144 20.7633 33.898 [...]
optimal.1000 = 24.6405 21.5385 24.6629 27.2982 34.1979 41.2304 41.6266 [...]
\end{verbatim}
We defined filtering and smoothing errors for \verb!prior! and \verb!optimal! mutations, with \verb!500! and \verb!1000! particles.
\end{itemize}

\subsection{Running \biips Test program in a terminal}

\paragraph{}
\biips Core and \biips Base are the two shared libraries needed by \biips Test. Before launching \biips Test program in a terminal, you first have to specify the shared libraries path in an environment variable. Type in a terminal\footnote{The text between angle brackets \texttt{<...>} must be replaced by its value. The text between curly brackets \texttt{\{...\}} indicates a choice between several values.}:
\begin{verbatim}
cd <BIIPS_PATH>/BiipsTest
export LD_LIBRARY_PATH=../BiipsCore/{Release|Debug}:../BiipsBase/{Release/Debug}
\end{verbatim}

\paragraph{}
You have to choose between Debug and Release versions.

\paragraph{}
Then, to launch \biips Test using model \verb=ID=, type in a terminal\footnote{The bracketed text is optional: \texttt{[<OPTION>]...} stands for several options.}:
\begin{verbatim}
{Release|Debug}/BiipsTest --model-id=<ID> [<OPTION>]...
\end{verbatim}

\paragraph{}
Several configuration files can be found in the folder \texttt{BiipsTest/bench}. They are generated by \texttt{run\_bench\_all} Matlab function located in the same folder. Type in a terminal:
\begin{verbatim}
{Release|Debug}/BiipsTest --cfg=bench/<CONFIG_FILE> --n-smc=<N> [<OPTION>]...
\end{verbatim}

\paragraph{}
In addition, \texttt{BiipsTest.sh} shell script runs \biips Test for all \texttt{.cfg} files in \texttt{BiipsTest/bench} directory. You need not to specify the environment variable using the script. Type in a terminal:
\begin{verbatim}
cd <BIIPS_PATH>/BiipsTest
./BiipsTest.sh {Debug|Release} <N> [<OPTION>]...
\end{verbatim}
to run Debug/Release \texttt{BiipsTest} binary executable with option \verb!--n-smc=N!.

\paragraph{}
Current \biips Test implementation can only check errors of \verb!model-id=1! or \verb!2!.

\paragraph{}
The program follows the general work-flow presented in figure \ref{fig:biipsttest}.


\begin{figure}[htbp]
%\centering
\begin{center}
\tikzstyle{every node}=[font=\footnotesize]
\tikzstyle{input}=[rectangle, rounded corners,
                                    thick,
                                    text width=3.5cm,
                                    draw=blue!80,
                                    fill=blue!20,
                                    text centered,
                                    %font=\large,
                                    ]
\tikzstyle{processing}=[rectangle, rounded corners,
                                    thick,
                                    text width=3.5cm,
                                    draw=green!80,
                                    fill=green!20,
                                    text centered,
                                    %font=\large,
                                    ]
\tikzstyle{output}=[rectangle, rounded corners,
                                                thick,
                                                text width=3.5cm,
                                                draw=orange!80,
                                                fill=orange!25,
                                                text centered,
                                    %font=\large,
]
% Everything is drawn on underlying gray rectangles with
% rounded corners.
\tikzstyle{background}=[rectangle,
                                                fill=gray!20,
                                                inner sep=0.2cm,
                                                rounded corners=5mm]
\tikzstyle{background_dark}=[rectangle,
                                                fill=gray!40,
                                                inner sep=0.4cm,
                                                rounded corners=5mm]
\tikzstyle{background_out}=[rectangle,
                                                fill=gray!20,
                                                inner sep=0.6cm,
                                                rounded corners=5mm]

\begin{tikzpicture}[node distance=2.5cm,auto,>=latex']

%Model
\node (model) [] {\textbf{Model:}};
\node (or1) [below of=model, node distance=0.8cm] {OR};
\node (read_param) [input, left of=or1] {Read parameters\\in configuration file};
\node (default_param) [processing, right of=or1] {Set default\\parameters};
\begin{pgfonlayer}{background}
\node (model_bg) [background, fit= (model) (read_param) (default_param)] {};
\end{pgfonlayer}

%Data
\node (data) [below of = model_bg, node distance=2cm] {\textbf{Data:} \texttt{step=0}};
\node (or2) [below of=data, node distance=0.8cm] {OR};
\node (read_data) [input, left of=or2] {Read data\\in configuration file};
\node (sample_data) [processing, right of=or2] {Sample data\\(using \texttt{data-rng-seed} option)};
\begin{pgfonlayer}{background}
\node (data_bg) [background, fit= (data) (read_data) (sample_data)] {};
\end{pgfonlayer}
\path[->] (model_bg) edge[thick] (data_bg);

%Bench
\node (bench) [below of=data_bg, node distance=2.1cm] {\textbf{Bench:} \texttt{step=1}};
\node (or3) [below of=bench, node distance=0.8cm] {OR};
\node (read_bench) [input, left of=or3] {Read benchmarks\\in configuration file\\(filtering/smoothing)};
\node (compute_bench) [processing, right of=or3] {Compute benchmarks\\if bench implemented\\(filtering/smoothing)};
\begin{pgfonlayer}{background}
\node (bench_bg) [background, fit=(bench) (read_bench) (compute_bench)] {};
\end{pgfonlayer}
\path[->] (data_bg) edge[thick] (bench_bg);

\node (particles) [below of=bench_bg, node distance=2.5cm] {\textbf{For each particles number in \texttt{particles} option}};

\node (mutation) [below of=particles, node distance=1cm] {\textbf{For each mutation in \texttt{mutations} option}};

%SMC
\node (smc) [below of=mutation, node distance=0.8cm] {\textbf{SMC:} \texttt{step=2}};
\node (run_smc) [processing, below of=smc, node distance=0.8cm] {Run \texttt{n-smc} SMC algorithms};
\node (or4) [below of=run_smc, node distance=2cm] {OR};
\node (smc_n_eq1) [output, left of=or4, text width=4.5cm, node distance=3cm] {\textbf{Case \texttt{n-smc=1}:}\\
Use \texttt{interactive}, \texttt{show-plots}, \texttt{plot-file}, \texttt{smc-rng-seed} options\\
-if \texttt{verbose=1}:\\
~~Display progress bar of the run\\
-if \texttt{verbose=2}:\\
~~Complete display};
\node (smc_n_gt1) [output, right of=or4, text width=4.5cm, node distance=3cm] {\textbf{Case \texttt{n-smc>1}:}\\
Use time initialized seed\\
-if \texttt{verbose=1}:\\
~~Display global progress bar\\
-if \texttt{verbose=2}:\\
~~Display progress bar of each run};
\node (smc_error) [processing, below of=or4, node distance=2.3cm, text width=4.5cm] {Compute errors vs benchmarks\\if available\\(filtering/smoothing)};
\path[->] (run_smc) edge[thick,style={bend right}] (smc_n_eq1);
\path[->] (run_smc) edge[thick,style={bend left}] (smc_n_gt1);
\path[->] (smc_n_eq1) edge[thick,style={bend right=35}] (smc_error);
\path[->] (smc_n_gt1) edge[thick,style={bend left=35}] (smc_error);

%Check
\node (check) [below of=smc_error, node distance=1.8cm] {\textbf{Check:} \texttt{step=3}};
\node (read_errors) [input, below of=check, node distance=0.8cm] {Read reference errors\\in configuration file\\(filtering/smoothing)};
\node (or5) [below of=read_errors, node distance=1.8cm] {OR};
\node (check_n_eq1) [processing, left of=or5, text width=4.5cm, node distance=3cm] {\textbf{Case \texttt{n-smc=1}:}\\
-Compute 1-(\texttt{reject-level})\\
quantile of the errors\\
(filtering/smoothing)\\
-Check error < quantile};
\node (check_n_gt1) [processing, right of=or5, text width=4.5cm, node distance=3cm] {\textbf{Case \texttt{n-smc>1}:}\\
-Compute Kolmogorov-Smirnov statistic\\
(filtering/smoothing)\\
-Check p-value > \texttt{reject-level}};
\path[->] (read_errors) edge[thick,style={bend right}] (check_n_eq1);
\path[->] (read_errors) edge[thick,style={bend left}] (check_n_gt1);

\begin{pgfonlayer}{background}
\node (particles_bg) [background_out, fit= (particles) (smc_n_eq1) (smc_n_gt1) (check_n_eq1) (check_n_gt1)] {};
\end{pgfonlayer}
\path[->] (bench_bg) edge[thick] (particles_bg);
\begin{pgfonlayer}{background}
\node (mutation_bg) [background_dark, fit= (mutation) (smc_n_eq1) (smc_n_gt1) (check_n_eq1) (check_n_gt1)] {};
\end{pgfonlayer}
\begin{pgfonlayer}{background}
\node (smc_bg) [background, fit=(smc) (smc_n_eq1) (smc_n_gt1) (smc_error)] {};
\node (check_bg) [background, fit= (check) (check_n_eq1) (check_n_gt1)] {};
\end{pgfonlayer}
\path[->] (smc_bg) edge[thick] (check_bg);

\end{tikzpicture}
\caption{\biips Test program general flowchart. Inputs are in blue, processing in green and display in orange.}
\label{fig:biipsttest}
\end{center}
\end{figure}


% The \texttt{BiipsTest} executable program has command-line arguments:
% \begin{itemize}
% \item 1st argument: an integer identifying the test to be executed. Three tests are currently available:
%   \begin{itemize}
%   \item Test 1: a linear gaussian 1D HMM\footnote{Hidden Markov Model} model compared with Kalman filter estimates
%   \item Test 2: a linear gaussian N-dimensional HMM model compared with Kalman filter estimates
%   \item Test 3: a Binomial success probability parameter estimation with Beta prior
%   \end{itemize}
% 
% \item 2nd argument: a data file name containing the inputs of the test. The data files must be in the \texttt{BiipsTest/data} subdirectory. Four template data files are provided:
%   \begin{itemize}{}{}
%   \item \texttt{data.dat} is for Test 1
%   \item \texttt{data\_1d.dat} is a 1D model for Test 2 (the same as for Test 1)
%   \item \texttt{data\_4d.dat} is a 4D model for Test 2 (representing 2 position dimensions and 2 velocity dimensions)
%   \item \texttt{data\_beta.dat} is for Test 3
%   \end{itemize}
%   See annex \ref{testsinput} for the details of each file.
% 
% \item 3rd (optional) argument: the \texttt{\textminus\textminus noprompt} option. It removes questions to the user during the test.
% 
% \item 4th (optional) argument: the \texttt{\textminus\textminus noplot} option. It removes showing the plots during the test.
% 
% \item 5th (optional) argument: the \texttt{\textminus\textminus prior} option. It runs the SMC algorithm sampling the particles according to the prior, \ie{} no advanced conjugate sampling methods.
% \end{itemize}
% Just write any text (\eg{} \texttt{\textminus} alone) instead of the optional options not to use them.


\subsection{Running/Debugging \biips Test in Eclipse}
At the first launch of \biips Test program with the \emph{Run/Debug} button, in the \emph{Select Preferred Launcher} window:
\begin{quote}
Tick \emph{use configuration specific settings}.\\
Select \emph{Standard Create Process Launcher}.\\
Click \emph{OK}.
\end{quote}
To edit the command-line arguments afterwards, follow the instructions:
\begin{quote}
Select \biips Test project.\\
Click the top menu \emph{Run/Debug $\rightarrow$ Run/Debug Configurations}.\\
Edit the program arguments in the \emph{Arguments} tab.
\end{quote}
Debug mode opens the Debug perspective of Eclipse and allows you to define breakpoints, run the program step by step and watch variable values.

% \paragraph{Example:} 
%  \begin{itemize}
%   \item \verb=1 data.dat --noprompt= will launch test 1 with input file \texttt{data.dat} and without prompting.
%   \item \verb=2 data_4d.dat --noprompt - --prior= will launch test 2 with input file \verb=data_4d.dat=, without prompting and using the prior distribution sampling method.
%  \end{itemize}
% 
% 
% \paragraph{}
% Ouput files are in the \texttt{BiipsTest/results} subdirectory.


% \section{Synchronizing}
% 
% 
% First, install the \textbf{Subclipse plugin:} by following the instructions in Eclipse:
% \begin{quote}
% Click the top menu \emph{Help $\rightarrow$ Install new software}. \\
% In the \emph{Work with} field, type: \verb=http://subclipse.tigris.org/update_1.6.x= \\
% Click \emph{Add} \\
% Select all the features (Core SVNKit Library, Optional JNA library, Subclipse). \\
% Click \emph{Next}, \emph{Next}, \emph{Accept}, \emph{Finish}. \\
% Answer \emph{OK} to the \emph{Security warning} message. \\
% Restart Eclipse.
% Answer \emph{OK} to the messages.
% \end{quote}
% Then, edit the preferences
% \begin{quote}
% Click the top menu\emph{ Window $\rightarrow$ Preferences $\rightarrow$ Team $\rightarrow$ SVN}. \\
% Select \emph{SVNKit} in \emph{SVN interface}.
% \end{quote}

% Window -> open Perspective -> team Synchronizing
% Left view, click the synchronize button
% Select SVN, Next, select all projects, finish
% Enter SSH credentials
% Use private key authentication, select the private key (<home>/.ssh/id\_rsa)
% Yes, ok 

% \section{Coding rules}



%%%%%%%%%%%%%%%%%%%%%%%%%%%%%%%%%%%%%%%%%%%%%%%%%%%%%%%%%%%%%%%%%%%%%%%%%%%%%%%%%%%%%%%%%%%%%%%%%%%%%%%%%%
%%
%% Chapter : Tasks
%%
%%%%%%%%%%%%%%%%%%%%%%%%%%%%%%%%%%%%%%%%%%%%%%%%%%%%%%%%%%%%%%%%%%%%%%%%%%%%%%%%%%%%%%%%%%%%%%%%%%%%%%%%%%
\chapter{To do}
\label{todo}

\section{Tasks}


\begin{enumerate}
 \item Exceptions handling
 \item Add distributions, functions, node samplers, tests, etc.
 \item BUGS language compiler: \\
  See section \ref{jags_compiler} for a study of JAGS compiler and its re-usability.
 \item Text Interface
 \item Replace SMCTC
 \item GPU\footnote{Graphic Processing Unit} Parallelization GPU using CUDA\footnote{Compute Unified Device Architecture}: \\
We will use CUDA to parallelize the particles mutation step. Important performance growths for SMC algorithms have been observed \cite{lee_utility_2009}. Indeed, the mutation step follows the SIMD\footnote{Same Instruction, Multiple Data} framework which is perfectly suited for GPU parallelization. We already have a desktop computer equipped with a NVidia GeForce GTX 460 GPU. This device is compatible with CUDA 2.1 which supports C++.\\
This task is not a priority as the functionality of the software comes before its performance in our objectives. However, this remains a marketing argument. A CUDA training course is planned on November, 9th.
 \item Graphical Interface
 \item Web site
 \item Port to other platform configurations (Windows, Mac)
 \item Technical report, scientific article
\end{enumerate}


\section{JAGS compiler}
\label{jags_compiler}

\paragraph{}
JAGS uses a scanner generated by Flex in the \texttt{scanner.cc} source file. The latter was generated from the \texttt{scanner.ll} definition file. See annex \ref{bisonflex} for more details about Flex\footnote{Fast Lexical analyzer} and Bison.

\paragraph{}
This scanner is used by the parser generated by Bison in the \texttt{parser.cc} source file, from the \texttt{parser.yy} grammar definition file. This source code defines a \texttt{parse\_bugs} function which analyzes the file describing a graphical model in BUGS language.

\paragraph{}
The aim of the actions defined in the grammar is to create a \texttt{ParseTree} object. Then a \texttt{Compiler} object is responsible for transforming it in a \texttt{BUGSModel} object. The latter is the one which will contain the node instances of the graphical model.

\begin{figure}[h!]
\begin{center}
\tikzstyle{input}=[rectangle, rounded corners,
                                    thick,
                                    text width=2.5cm,
                                    draw=blue!80,
                                    fill=blue!20,
                                    text centered,
                                    font=\large,
                                    ]
\tikzstyle{processing}=[rectangle, rounded corners,
                                    thick,
                                    text width=2.5cm,
                                    draw=green!80,
                                    fill=green!20,
                                    text centered,
                                    font=\large,
                                    ]
\tikzstyle{output}=[rectangle, rounded corners,
                                                thick,
                                                text width=2.5cm,
                                                draw=orange!80,
                                                fill=orange!25,
                                                text centered,
                                    font=\large,]
\begin{tikzpicture}[node distance=3.5cm,auto,>=latex']
\node (code) [input]   {BUGS language model};
\node (parser)   [processing,right of= code] {\texttt{parse\_bugs}};
\node (parsetree)   [output,right of= parser] {\texttt{ParseTree}};
\node (compiler)   [processing,right of=parsetree] {\texttt{Compiler}};
\node (model)   [output,right of=compiler] {\texttt{BUGSModel}};
 \path[->] (code) edge[thick] (parser);
 \path[->] (parser) edge[thick] (parsetree);
 \path[->] (parsetree) edge[thick] (compiler);
 \path[->] (compiler) edge[thick] (model);
\end{tikzpicture}
\label{fig:briques}
\caption{JAGS: Graphical model compilation process}
\end{center}
\end{figure}


\subsection{Re-usability of the ParseTree output}
The \texttt{ParseTree} class is completely independent of the rest of JAGS code, hence we can directly use it to build a \biips{} model.

\paragraph{}
A \texttt{ParseTree} object represents a BUGS language definition. Its \textquotedblleft class\textquotedblright{} is defined by a \texttt{TreeClass} member
\begin{lstlisting}
enum TreeClass { 
    P_VAR, P_RANGE, P_BOUNDS, P_DENSITY, P_LINK, P_COUNTER, 
    P_VALUE, P_STOCHREL, P_DETRMREL, P_FOR,  P_FUNCTION, P_RELATIONS,
    P_VECTOR, P_ARRAY, P_DIM, P_LENGTH, P_SUBSET
};
\end{lstlisting}

It also has a name (for \verb=P_VAR=, \verb=P_COUNTER=, \verb=P_FUNCTION=, \verb=P_DISTRIBUTION=, \verb=P_LINK=, or \verb=P_ARRAY= \textquotedblleft class\textquotedblright{}), a value (for  \verb=P_VALUE= \textquotedblleft class\textquotedblright{}), parameters (which are other \texttt{ParseTree} objects) to which it takes ownership, its line number in the model definition. Consequently, there exist a top-level \texttt{ParseTree} object that owns all the others.

\paragraph{}
Those parts of the code are completely isolated from the rest and are, hence, easily reusable. These are the parts of JAGS code we could reuse in \biips{}:
\begin{itemize}
 \item \texttt{src/lib/compiler/scanner.cc}
 \item \texttt{src/lib/compiler/parser.cc}
 \item \texttt{src/lib/compiler/remap.h}
 \item \texttt{src/lib/compiler/parser.h}
 \item \texttt{src/lib/compiler/parser\_extra.h}
 \item \texttt{src/lib/compiler/ParseTree.cc}
 \item \texttt{src/include/compiler/ParseTree.h}
\end{itemize} 

\paragraph{}
We would therefore reprogram the creation of \biips{} objects from the \texttt{ParseTree} object (\textit{i.e.} the \texttt{Compiler} class).


\subsection{Re-usability of the Compiler class code}
It would be ideal to be able to reuse the \texttt{Compiler} class (1135 lines) to generate a \biips{} model instead of a \texttt{BUGSModel} JAGS object. But this part of the code is wholly dependent of the rest of JAGS code, which makes its re-usability more complicated. We can however take ideas on it.





%%%%%%%%%%%%%%%%%%%%%%%%%%%%%%%%%%%%%%%%%%%%%%%%%%%%%%%%%%%%%%%%%%%%%%%%%%%%%%%%%%%%%%%%%%%%%%%%%%%%%%%%%%
%%--------------------------------------------------------------------------------------------------------
%% Appendices
%%--------------------------------------------------------------------------------------------------------
%%%%%%%%%%%%%%%%%%%%%%%%%%%%%%%%%%%%%%%%%%%%%%%%%%%%%%%%%%%%%%%%%%%%%%%%%%%%%%%%%%%%%%%%%%%%%%%%%%%%%%%%%%
\appendix

\chapter{Flex and Bison}
\label{bisonflex}

A compiler is composed of a scanner (lexical analyzer) and a parser (syntactic analyzer). Both programs can be generated by tools such as:
\begin {itemize}
 \item Flex (GPL), a free implementation of Lex, a lexical analyzer generator.
 \item Bison (LGPL\footnote{GNU Lesser General Public License}), a free implementation of Yacc\footnote{Yet another compiler compiler}, a parser generator.
\end {itemize}

\section{The lexical analyzer}
Parses the source text into tokens defined by regular expressions.

\paragraph{Example :} C function divided into tokens
\begin{quote}
\verb=int            /*= keyword 'int' \verb=*/= \\
\verb=square (int x) /*= identifier, open-paren, keyword 'int', identifier, close-paren \verb=*/= \\
\verb={              /*= open-brace *\verb=*/= \\
\verb=return x * x;  /*= keyword 'return', identifier, asterisk, identifier, semicolon \verb=*/= \\
\verb=}              /*= close-brace \verb=*/= 
\end{quote}

\paragraph{}
The description of the scanner is in the form of pairs of regular expressions and C code, called rules. Flex generates as output a C source file that defines a routine \texttt{yylex()}. It analyzes the occurrences of regular expressions of its input. At each detection, the corresponding C code is executed.


\section{The syntactic analyzer}
Reads a sequence of tokens and decides whether it corresponds to a syntax defined by the grammar.

\paragraph{}
The grammar is composed of symbols:
\begin{itemize}
\item nonterminal symbols (or groupings), conventionally represented by a lowercase identifier. \eg{} : \texttt{expr}, \texttt{stmt}, \ldots
\item terminal symbols (or token types), represented by
  \begin{itemize}
  \item an uppercase identifier. \eg{} : \texttt{INTEGER}, \texttt{IDENTIFIER}, \texttt{IF}, \texttt{RETURN}, \ldots
  \item a C character constant. \eg{} : \texttt{'('}, \texttt{')'}, \texttt{'+'}, \ldots
  \item a constant C string.
  \end{itemize}
\end{itemize}

Each nonterminal symbol must match a rule.

\paragraph{Example :} C \texttt{return} statement rule
\begin{quote}
\begin{verbatim}
stmt: RETURN expr ';'
      ;
\end{verbatim}
\end{quote}

\paragraph{}
Each grammar rule is assigned an action, \ie{} a set of instructions to be executed when the C syntax is identified. These actions are written in braces after the rule.

\paragraph{Example :} Rule stating that an expression may be the sum of two subexpressions
\begin{quote}
\begin{verbatim}
expr: expr '+' expr { $$ = $1 + $3; }
      ;
\end{verbatim}
\end{quote}

\paragraph{}
Here, the \texttt {\$\$}, \texttt{\$1} and \texttt{\$3} signs represent the semantic values of the sum expression, and the two sub-expressions.

\paragraph{}
Bison generates source C (or C++) code of the parser from a grammar file definition. The code defines a function \texttt{yyparse} which implements this grammar. The Bison grammar file take the general form:
\begin{quote}
\begin{verbatim}
%{
Prologue
%}
Bison declarations
%%
Grammar rules
%%
Epilogue
\end{verbatim}
\end{quote}

\paragraph{}
The prologue declares types and variables used in the actions, we can also write preprocessor commands such as \texttt{\#include}. The lexical analyzer \texttt{yylex} and the error printer \texttt{yyerror} must be declared here.

\paragraph{}
The Bison declarations declare the names of the particular terminal and nonterminal symbols.

\paragraph{}
The grammar rules define how to construct each nonterminal symbol from its parts.

\paragraph{}
The epilogue can contain any code you want to use. Often the definitions of functions declared in the prologue go here.


\chapter{Codes}

\section{Simple example}
\label{miniex}

\lstset{language = C++, %
 basicstyle=\small\ttfamily, %
keywordstyle=\bfseries\color{RedViolet}, %
identifierstyle=,%
 commentstyle=\color{ForestGreen},%
stringstyle=\color{blue},%
%column=fixed,%
basewidth={0.5em,0.45em},%
numbers=left, numberfirstline=false, numberstyle=\tiny, stepnumber=10, numbersep=5pt}

\begin{lstlisting}
//                                               -*- C++ -*-
/*! \file t_Hmm1D_mini.cpp
 * \brief BiiPS minimal example
 * 
 * \author  $LastChangedBy: todeschi $
 * \date    $LastChangedDate: 2011-03-23 17:23:46 +0100 (mer. 23 mars 2011) $
 * \version $LastChangedRevision: 126 $
 * Id:      $Id: t_Hmm1D_mini.cpp 126 2011-03-23 16:23:46Z todeschi $
 *
 *
 * BiiPS example : Hidden Markov Model linear gaussian 1D
 * ======================================================
 * x[0] --> x[1] --> ... --> x[t] --> ...
 *           |                |
 *           v                v
 *          y[1]     ...     y[t]     ...
 *
 *          x[0] ~ Normal(mean_x0, var_x0)
 * x[t] | x[t-1] ~ Normal(x[t-1], var_x) for all t>0
 *   y[t] | x[t] ~ Normal(x[t], var_y)  for all t>0
 *
 */

#include "BiipsCore.hpp"
#include "BiipsBase.hpp"

int main(int argc, char* argv[])
{

  using namespace Biips;

 /*==================================*
  *                                  *
  * Instantiate the model with BiiPS *
  *                                  *
  *==================================*/

  // Define the data :
  //------------------
  // Final time :
  Size t_max = 20;
  // Initial mean of X[0]
  Scalar mean_x0_val = 0;
  // Variance of X[0] :
  Scalar var_x0_val = 1;
  // Variance of X[t] | X[t-1] :
  Scalar var_x_val = 1;
  // Variance of Y[t] | Y[t] :
  Scalar var_y_val = 0.5;

  // Declare FunctionTable and a DistributionTable :
  //------------------------------------------------
  FunctionTable func_tab;
  DistributionTable dist_tab;

  // Load the Base module :
  //-----------------------
  loadBaseModule(func_tab, dist_tab);

  // Declare the Graph object :
  //---------------------------
  Graph graph;

  // Add the constant nodes to the graph :
  //--------------------------------------
  NodeId mean_x0 = graph.AddConstantNode(MultiArray(mean_x0_val));
  NodeId var_x0 = graph.AddConstantNode(MultiArray(var_x0_val));
  NodeId var_x = graph.AddConstantNode(MultiArray(var_x_val));
  NodeId var_y = graph.AddConstantNode(MultiArray(var_y_val));

  // Create stochastic NodeId collections to store X and Y's components identifiers :
  //---------------------------------------------------------------------------------
  Types<NodeId>::Array x(t_max+1);
  Types<NodeId>::Array y(t_max);

  // Create a NodeId array to handle the parents of each stochastic node :
  //----------------------------------------------------------------------
  Types<NodeId>::Array params(2);

  // Add X_0 stochastic node to the graph :
  //---------------------------------------
  params[0] = mean_x0;
  params[1] = var_x0;
  x[0] = graph.AddStochasticNode(P_SCALAR_DIM, dist_tab["dnormvar"], params, false);

  // Add the other stochastic nodes to the graph :
  //----------------------------------------------
  for (Size t=1; t<t_max+1; ++t)
  {
    // Add X[t]
    params[0] = x[t-1];
    params[1] = var_x;
    x[t] = graph.AddStochasticNode(P_SCALAR_DIM, dist_tab["dnormvar"], params, false);

    // Add Y[t]
    params[0] = x[t];
    params[1] = var_y;
    y[t-1] = graph.AddStochasticNode(P_SCALAR_DIM, dist_tab["dnormvar"], params, true);
  }

  // Build the graph :
  //------------------
  graph.Build();

  // Generate data :
  //----------------
  // Declare a random number generator, initialized with an integer seed
  Rng my_rng(0);

  // Sample values according to the stochastic nodes prior distribution
  NodeValues gen_values = graph.SampleValues(&my_rng);

  // Set the observations values according to the generated values
  graph.SetObsValues(gen_values);


 /*==================================*
  *                                  *
  *     Run the SMC algorithm        *
  *                                  *
  *==================================*/

  // Define the number of particles :
  //---------------------------------
  Size nb_particles = 1000;

  // Declare the SMCSampler object :
  //--------------------------------
  SMCSampler sampler(nb_particles, &graph, &my_rng);
  sampler.SetResampleParams(SMC_RESAMPLE_STRATIFIED, 0.5);

  // Initialize the SMCSampler object :
  //-----------------------------------
  sampler.Initialize();

  // Declare and configure a ScalarAccumulator object :
  //---------------------------------------------------
  ScalarAccumulator stats_acc;
  stats_acc.AddFeature(MEAN);
  stats_acc.AddFeature(VARIANCE);

  // Declare Scalar arrays to store the posterior mean and variance estimates :
  //---------------------------------------------------------------------------
  Types<Scalar>::Array x_est_PF(t_max+1);
  Types<Scalar>::Array x_var_PF(t_max+1);

  // Iterate the SMC algorithm and extract summary statistics :
  //-----------------------------------------------------------
  for (Size t=0; t<t_max+1; ++t)
  {
    // Iterate the SMC algorithm : resample (if needed) and sample one node of the sequence
    sampler.Iterate();

    // Accumulate particles corresponding to the last updated node
    sampler.Accumulate(x[t], stats_acc);

    // extract summary statistics of the filtering density
    x_est_PF[t] = stats_acc.Mean();
    x_var_PF[t] = stats_acc.Variance();
  }

  return 0;

}
\end{lstlisting}
% 
% \chapter{Tests input files}
% \label{testsinput}
% \section{Test 1}
% 
% \begin{table*}[h!]
% \begin{tabular}{l|l}
% 20         & Final time : t\_max \\
% 0         &  Initial mean of x[0] : mu\_x0 \\
% 1        &   Variance of x[0] : sig\_x0 \\
% 1        &   Mean multiplicator of x[t] | x[t-1] : a\_x \\
% 0        &   Mean second member of x[t] | x[t-1] : b\_x \\
% 1        &   Variance of x[t] | x[t-1] : sig\_x \\
% 1         &  Mean multiplicator of y[t] | x[t] : a\_y \\
% 1         &  Variance of y[t] | x[t] : sig\_y \\
% 4589      &  Random number generator seed : rng\_seed \\
% 1000      &  Number of particles : N \\
% results.dat & Results data file \\
% plot.pdf   & Plot file \\
% gnuplot.plt & Gnuplot file
% \end{tabular}
% \caption{\texttt{data.dat}}
% \end{table*}
% 
% \section{Test 2}
% 
% \begin{table*}[h!]
% \begin{tabular}{l|l}
% 1                 & State x dimension : dim\_x \\
% 1                & Control u dimension : dim\_u \\
% 1                 & Observation y dimension : dim\_y \\
% 20                & Final time : t\_max \\
% 0                & Initial state mean vector : mu\_x0 \\
% 1                & Initial state covariance matrix : P0 \\
% 1                & State transition matrix : F \\
% 0                & Control transition matrix : B \\
% 0                & Control vector : u \\
% 1                & State covariance matrix : Q \\
% 1                & Observation transition matrix : H \\
% 1                & Observation covariance matrix : R \\
% 4589            &         Random number generator seed : rng\_seed   \\          
% 1000              &     Number of particles : N     \\          
% results\_1d.dat  & Results data file      \\             
% plot\_1d.pdf     & Plot file \\
% gnuplot\_1d.plt   & Gnuplot file
% \end{tabular}
% \caption{\texttt{data\_1d.dat}}
% \end{table*}
% 
% 
% \begin{table*}[h!]
% \begin{tabular}{l|l}
% 4   &   State x dimension : dim\_x\\
% 1  & Control u dimension : dim\_u\\
% 2  & Observation y dimension : dim\_y\\
% 10 &  Final time : t\_max \\
% 0  & Initial state mean vector : mu\_x0\\
% 0 & \\
% 1 & \\
% 1 & \\
% 1 0 0 0   &Initial state covariance matrix : P0\\
% 0 1 0 0 & \\
% 0 0 1 0 & \\
% 0 0 0 1 & \\
% 1 0 1 0  & State transition matrix : F\\
% 0 1 0 1 & \\
% 0 0 1 0 & \\
% 0 0 0 1&  \\
% 0  & Control transition matrix : B \\
% 0 & \\
% 0 & \\
% 0 & \\
% 0  & Control vector : u \\
% 1 0 0 0  & State covariance matrix : Q \\
% 0 1 0 0 & \\
% 0 0 1 0 & \\
% 0 0 0 1 & \\
% 1 0 0 0  & Observation transition matrix : H \\
% 0 1 0 0 & \\
% 1 0  & Observation covariance matrix : R \\
% 0 1 & \\
% 4535  & Random number generator seed : rng\_seed    \\
% 1000 &  Number of particles : N           \\    
% results\_4d.dat &  Results data file     \\              
% plot\_4d.pdf &  Plot file \\
% gnuplot\_4d.plt   & Gnuplot file
% \end{tabular}
% \caption{\texttt{data\_4d.dat}}
% \end{table*}
% 
% \section{Test 3}
% 
% \begin{table*}[h!]
% \begin{tabular}{l|l}
% 20   &   Final time : t\_max \\
% 2     &   Prior parameter alpha : alpha \\
% 2     &  Prior parameter beta : beta \\
% 10        &   Number of trials : n \\
% 4589       &          Random number generator seed : rng\_seed   \\
% 1000              &   Number of particles : N   \\            
% results\_beta.dat   &  Results data file  \\                 
% plot\_beta.pdf      &  Plot file \\
% gnuplot\_beta.plt   &  Gnuplot file
% \end{tabular}
% \caption{\texttt{data\_beta.dat}}
% \end{table*}




%%%%%%%%%%%%%%%%%%%%%%%%%%%%%%%%%%%%%%%%%%%%%%%%%%%%%%%%%%%%%%%%%%%%%%%%%%%%%%%%%%%%%%%%%%%%%%%%%%%%%%%%%%
%%
%% Glossary
%%
%%%%%%%%%%%%%%%%%%%%%%%%%%%%%%%%%%%%%%%%%%%%%%%%%%%%%%%%%%%%%%%%%%%%%%%%%%%%%%%%%%%%%%%%%%%%%%%%%%%%%%%%%%
\chapter*{Glossary}
\addcontentsline{toc}{chapter}{Glossary}

\begin{list}{}{}
 \item[\textbf{Class:}] a class in C++ is a user-defined type that aggregates \textbf{members} (data fields of other types) and that defines its own \textbf{function members} (or \textbf{methods}), according to the principle of encapsulation. This design is specific to object-oriented programming. An instance of a class is called an \textbf{object}. An \textbf{abstract} class has at least one \textbf{pure virtual} function member. It only consists in a common interface for other \textbf{concrete} classes which will implement those pure virtual functions. Abstract classes can not be instantiated.


%  \item[\textbf{Factory:}] 


 \item[\textbf{Shared pointer:}] a smart pointer with shared ownership semantics used to store dynamically allocated variables. The ownership of the variables is shared between several pointers thanks to a counter based system. This way, the release of the allocated memory is automatic when the counter is null. Not only this avoids memory leaks, but this also provides a smart storage management avoiding duplications. 


 \item[\textbf{\texttt{namespace}:}] a C++ keyword defining a scope for names of C++ declarations.
\begin{verbatim}
namespace Biips
{
 class Graph;
}
\end{verbatim}
Outside the namespace scope, all names residing in it will be called using the namespace name before the \verb=::= scope resolution symbol as prefix.
\begin{verbatim}
Biips::Graph graph;
\end{verbatim}
The \verb=using namespace= command allows us to use the names residing in the namespace without a prefix.
\begin{verbatim}
using namespace Biips;
Graph graph;
\end{verbatim}
Namespaces can be aliased.
\begin{verbatim}
namespace ublas = boost::numeric::ublas;
\end{verbatim}


 \item[\textbf{Polymorphism:}] is a principle of object-oriented programming. Basically, this corresponds to the idea that the execution of a portion of code depends on the type of data it applies on. One specific way of doing polymorphism in C++ is the inheritance of virtual member functions from a base class.


 \item[\textbf{\texttt{static}:}] a C++ keyword particularly used to declare a static member or function member. A static member or function member is owned by the class itself and is not associated to any instance (object). Modification of a static member applies to the whole class. Static member functions can not access non static members.


 \item[\textbf{\texttt{typedef}:}] a C++ keyword that defines aliases of other types.
\begin{verbatim}
typedef double Scalar;
\end{verbatim}


 \item[\textbf{\texttt{virtual}:}] a C++ keyword declaring a virtual function member. All derived classes can redefine their own implementation of these function members. Basically, a call to these virtual functions will result in the execution of the derived implementation even if the objects are stored in a base class object. This type of link is called \textbf{dynamic} because the executed code is only known at run-time. \textbf{Pure virtual} functions are only declared, they are only implemented in derived classes.

\end{list}




%%%%%%%%%%%%%%%%%%%%%%%%%%%%%%%%%%%%%%%%%%%%%%%%%%%%%%%%%%%%%%%%%%%%%%%%%%%%%%%%%%%%%%%%%%%%%%%%%%%%%%%%%%
%%--------------------------------------------------------------------------------------------------------
%% Listes des figures, tableaux, algorithmes
%%--------------------------------------------------------------------------------------------------------
%%%%%%%%%%%%%%%%%%%%%%%%%%%%%%%%%%%%%%%%%%%%%%%%%%%%%%%%%%%%%%%%%%%%%%%%%%%%%%%%%%%%%%%%%%%%%%%%%%%%%%%%%%
% \listoffigures
% \addcontentsline{toc}{chapter}{Table des figures}
% \addcontentsline{toc}{chapter}{List of Figures}

% \listoftables
% \addcontentsline{toc}{chapter}{Liste des tableaux}
% \addcontentsline{toc}{chapter}{Liste of Tables}

%\listofalgorithms
%\addcontentsline{toc}{chapter}{Liste des algorithmes}
%\addcontentsline{toc}{chapter}{List of Algorithms}

%%%%%%%%%%%%%%%%%%%%%%%%%%%%%%%%%%%%%%%%%%%%%%%%%%%%%%%%%%%%%%%%%%%%%%%%%%%%%%%%%%%%%%%%%%%%%%%%%%%%%%%%%%
%%--------------------------------------------------------------------------------------------------------
%% Bibliographie
%%--------------------------------------------------------------------------------------------------------
%%%%%%%%%%%%%%%%%%%%%%%%%%%%%%%%%%%%%%%%%%%%%%%%%%%%%%%%%%%%%%%%%%%%%%%%%%%%%%%%%%%%%%%%%%%%%%%%%%%%%%%%%%
\nocite{}
\bibliographystyle{plain}
\bibliography{biips_technical_report}


\end{document}
\endinput

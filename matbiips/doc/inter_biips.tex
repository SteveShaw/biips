\documentclass[11pt]{article}
\usepackage[english]{babel}
\usepackage[latin1]{inputenc}
\usepackage[dvips]{graphicx}
\usepackage{pstricks}
\usepackage{epsfig}
\usepackage{a4wide}

\usepackage{listings}
\usepackage{relsize}
\definecolor{dkgreen}{rgb}{0,0.6,0}
\definecolor{gray}{rgb}{0.8,0.8,0.8}
\definecolor{mauve}{rgb}{0.58,0,0.82}
\newcommand{\biips}{{\bf Biips\ }{}}
\newcommand{\matlab}{{\bf MATLAB\ }{}}
\lstset{ %
  framerule=0pt,
  basicstyle=\relsize{-2}\ttfamily,  % the size of the fonts that are used for the code
  backgroundcolor=\color{gray},  % choose the background color. You must add \usepackage{color}
  showspaces=false,               % show spaces adding particular underscores
  showstringspaces=false,         % underline spaces within strings
  showtabs=false,                 % show tabs within strings adding particular underscores
  %frame=single,                   % adds a frame around the code
  rulecolor=\color{black},        % if not set, the frame-color may be changed on line-breaks within not-black text (e.g. commens (green here))
  breakatwhitespace=false,        % sets if automatic breaks should only happen at whitespace
  keywordstyle=\color{blue},      % keyword style
  commentstyle=\color{dkgreen},   % comment style
  stringstyle=\color{mauve}
}




%%%%%%%%%%%%%%%%%%%%%%%%%%%%%%%%%%%%%%%%%%%%%%%%%%%%%%%%%%%%%%%%%%%%%%%%%%%%%
\begin{document}

\section{make\_console}
 \begin{lstlisting}[language=matlab]
   p = inter_biips('make_console')
 \end{lstlisting}
   
   \texttt{MAKE\_CONSOLE}  creates a new console and returns the id of the console in the global table of console. It
   is strongly recommended to store the number of the console in a variable, to be able later to communicate with Biips trough that
   console

\section{clear\_console}

 \begin{lstlisting}
   inter_biips('clear_console', id)
 \end{lstlisting}

  \texttt{CLEAR\_CONSOLE} suppress the console with the given \texttt{id} and free the memory associated

\section{clear\_all\_console}
 
 \begin{lstlisting}
   inter_biips('clear_all_console')
 \end{lstlisting}

  \texttt{CLEAR\_ALL\_CONSOLE} suppress all the consoles and free memory

\section{check\_model}

 \begin{lstlisting}
   inter_biips('check_model', id, filename)
 \end{lstlisting}
  
 \texttt{CHECK\_MODEL} verifies that the model given in the \texttt{filename} (with extension .bug) is consistent and associate it to the console \texttt{id}

\section{compile\_model}

 \begin{lstlisting}
   inter_biips('compile_model', id, vars, sample_data, seed) 
 \end{lstlisting}
   \begin{description}
   \setlength{\baselineskip}{0.1\baselineskip}
     \item[\texttt{id}] : the number of the console
     \item[\texttt{vars}] : a struct containing variables as fields, and the values in the model as values.
     \item[\texttt{sample\_data}] : a boolean flag indicating if data is given or sampled following the \texttt{data} section in the bug file
     \item[\texttt{seed}] : value of the seed used for random number generation
   
   \end{description}

   \texttt{COMPILE\_MODEL} compile the model previously furnished by the \texttt{CHECK\_MODEL} function. You can set up values of variables of 
   the model.


\section{get\_data}
   
 \begin{lstlisting}
   data = inter_biips('get_data', id) 
 \end{lstlisting}
 
  \texttt{GET\_DATA} retrieves all the data linked to a consoled with \texttt{id} and put it in the return value.


\section{load\_module}
 
 \begin{lstlisting}
   ok = inter_biips('load_module', name)
 \end{lstlisting}

  \texttt{LOAD\_MODULE} load the module \texttt{name} in Biips. Currently, it only works with \texttt{'basemod'} module. A 
  boolean flag is return if the operation is successful.

\section{verbosity}
 
 \begin{lstlisting}
  inter_biips('verbosity', level) 
 \end{lstlisting}

  \texttt{VERBOSITY} set the verbosity level of biips. \texttt{level} must belong to the set $\{0,1,2\}$

\section{get\_variables\_names}

 \begin{lstlisting}
  names = inter_biips('get_variables_names', id)
 \end{lstlisting}

 \texttt{GET\_VARIABLES\_NAMES} retrieves the names of variables of current model of the console \texttt{id} . the return value \texttt{names}
 is a cell of string


\end{document}

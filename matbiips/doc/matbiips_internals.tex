\documentclass[11pt,twoside]{article}
\usepackage[english]{babel}
\usepackage[latin1]{inputenc}
\usepackage[dvips]{graphicx}
\usepackage{pstricks}
\usepackage{epsfig}
\usepackage{a4wide}
\usepackage{listings}
\usepackage{relsize}
\definecolor{dkgreen}{rgb}{0,0.6,0}
\definecolor{gray}{rgb}{0.8,0.8,0.8}
\definecolor{mauve}{rgb}{0.58,0,0.82}
\newcommand{\biips}{{\bf Biips}{}}
\newcommand{\matlab}{{\bf MATLAB\ }{}}
\usepackage{hyperref}
\usepackage{lmodern}
\lstset{ %
  language=matlab,
  framerule=0pt,
  basicstyle=\relsize{-2}\ttfamily,  % the size of the fonts that are used for the code
  backgroundcolor=\color{gray},  % choose the background color. You must add \usepackage{color}
  showspaces=false,               % show spaces adding particular underscores
  showstringspaces=false,         % underline spaces within strings
  showtabs=false,                 % show tabs within strings adding particular underscores
  frame=single,                   % adds a frame around the code
  rulecolor=\color{black},        % if not set, the frame-color may be changed on line-breaks within not-black text (e.g. commens (green here))
  breakatwhitespace=false,        % sets if automatic breaks should only happen at whitespace
  keywordstyle=\color{blue},      % keyword style
  commentstyle=\color{dkgreen},   % comment style
  stringstyle=\color{mauve}
}



\begin{document}

\title{MatBiips internals reference guide}
\date{}

\maketitle

\tableofcontents

\section{introduction}

  Here, we have enumerated all the "internal" functions that one could call using the \texttt{matbiips} function. These functions are not intended
  to be used in a basic usage of \biips\ through the Matlab interface, but an user could rather use them in an aim of efficiency when he needs to
  do elementary works on the \biips\ model: e.g. change some values of variables, monitors variables, etc ...

\section{make\_console}
 \begin{lstlisting}
   p = matbiips('make_console')
 \end{lstlisting}

   \texttt{MAKE\_CONSOLE}  creates a new console and returns the id of the console in the global table of console. It
   is strongly recommended to store the number of the console in a variable, to be able later to communicate with \biips\  through that
   console

\section{clear\_console}

 \begin{lstlisting}
   matbiips('clear_console', id)
 \end{lstlisting}

  \texttt{CLEAR\_CONSOLE} suppresses the console with the given \texttt{id} and frees the associated memory

\section{clear\_all\_console}

 \begin{lstlisting}
   matbiips('clear_all_console')
 \end{lstlisting}

  \texttt{CLEAR\_ALL\_CONSOLE} suppresses all the consoles and frees memory

\section{check\_model}

 \begin{lstlisting}
   matbiips('check_model', id, filename)
 \end{lstlisting}

 \texttt{CHECK\_MODEL} verifies that the model given in the \texttt{filename} (with extension .bug) is consistent and links it to the console \texttt{id}

\section{compile\_model}

 \begin{lstlisting}
   matbiips('compile_model', id, vars, sample_data, seed)
 \end{lstlisting}
   \begin{description}
   \setlength{\baselineskip}{0.1\baselineskip}
     \item[\texttt{id}]: the number of the console
     \item[\texttt{vars}]: a struct containing variables as fields, and the values in the model as values.
     \item[\texttt{sample\_data}]: a boolean flag indicating if data is given or sampled following the \texttt{data} section in the bug file
     \item[\texttt{seed}]: value of the seed used for random number generation

   \end{description}

   \texttt{COMPILE\_MODEL} compiles the model previously furnished by the \texttt{CHECK\_MODEL} function. You can set up values of variables of
   the model.


\section{get\_data}

 \begin{lstlisting}
   data = matbiips('get_data', id)
 \end{lstlisting}

  \texttt{GET\_DATA} retrieves all the data linked to a console with \texttt{id} and put it in the return value.


\section{load\_module}

 \begin{lstlisting}
   ok = matbiips('load_module', name)
 \end{lstlisting}

  \texttt{LOAD\_MODULE} loads the module \texttt{name} in \biips. Currently, it only works with \texttt{'basemod'} module. A
  boolean flag is returned if the operation is successful.

\section{verbosity}

 \begin{lstlisting}
  matbiips('verbosity', level)
 \end{lstlisting}

  \texttt{VERBOSITY} sets the verbosity level of \biips . \texttt{level} must belong to the set $\{0,1,2\}$

\section{get\_variables\_names}

 \begin{lstlisting}
  names = matbiips('get_variable_names', id)
 \end{lstlisting}

 \texttt{GET\_VARIABLE\_NAMES} retrieves the names of the variables of the current model of the console \texttt{id}. The returned value \texttt{names}
 is a cell of strings containing the names of the variables

\section{set\_default\_monitors}

 \begin{lstlisting}
  names = matbiips('set_default_monitors', id)
 \end{lstlisting}

 \texttt{SET\_DEFAULT\_MONITORS} sets up the default filters for backward smoothing step for console model \texttt{id}

\section{set\_filter\_monitors}
 \begin{lstlisting}
  matbiips('set_filter_monitors', id, monitored_vars, lower, upper )
 \end{lstlisting}

   \begin{description}
   \setlength{\baselineskip}{0.1\baselineskip}
     \item[\texttt{id}]: the number of the console
     \item[\texttt{monitored\_vars}]: cell containing strings for variables to be monitored
     \item[\texttt{lower}]: cell containing lower bound for range monitoring
     \item[\texttt{upper}]: cell containing upper bound for range monitoring

   \end{description}

   \texttt{SET\_FILTER\_MONITORS} sets which variables must be monitored in the model. For instance to monitor x[1:10,1], one can use
 \begin{lstlisting}
  matbiips('set_filter_monitors', id, {'x'} ,  {1,1} , {10,1} )
 \end{lstlisting}

\section{set\_gen\_tree\_smooth\_monitors}

 \begin{lstlisting}
  matbiips('set_gen_tree_smooth_monitors', id, smooth_monitored_vars, lower, upper )
 \end{lstlisting}

   \begin{description}
   \setlength{\baselineskip}{0.1\baselineskip}
     \item[\texttt{id}]: the number of the console
     \item[\texttt{smooth\_monitored\_vars}]: cell containing strings for variables to be smooth monitored
     \item[\texttt{lower}]: cell containing lower bound for range smooth monitoring
     \item[\texttt{upper}]: cell containing upper bound for range smooth monitoring

   \end{description}

   \texttt{SET\_GEN\_TREE\_SMOOTH\_MONITORS} sets which variables must be smoothly monitored in the model. For instance to smoothly monitor x[1:10,1], one can use
 \begin{lstlisting}
  matbiips('set_gen_tree_smooth_monitors', id, {'x'} ,  {1,1} , {10,1} )
 \end{lstlisting}


\section{set\_backward\_smooth\_monitors}

 \begin{lstlisting}
  matbiips('set_backward_smooth_monitors', id, bw_smooth_monitored_vars, lower, upper )
 \end{lstlisting}

   \begin{description}
     \item[\texttt{id}]: the number of the console
     \item[\texttt{bw\_smooth\_monitored\_vars}]: cell containing strings for variables to be backward smoothly monitored
     \item[\texttt{lower}]: cell containing lower bound for range smooth monitoring
     \item[\texttt{upper}]: cell containing upper bound for range smooth monitoring

   \end{description}

   \texttt{SET\_BACKWARD\_SMOOTH\_MONITORS} sets which variables must be backward smoothly monitored in the model. For instance to backward smoothly monitor x[1:10,1], one can use
 \begin{lstlisting}
  matbiips('set_backward_smooth_monitors', id, {'x'} ,  {1,1} , {10,1} )
 \end{lstlisting}

\section{build\_smc\_sampler}

 \begin{lstlisting}
  matbiips('build_smc_sampler', id, prior_flag)
 \end{lstlisting}
  \texttt{BUILD\_SMC\_SAMPLER} builds the smc sampler associated to \texttt{id} console setting up possibly the prior method for sampling (\texttt{prior\_flag})

\section{is\_sampler\_built}

 \begin{lstlisting}
  ok = matbiips('is_sampler_built', id)
 \end{lstlisting}
  \texttt{IS\_SAMPLER\_BUILT} returns the \texttt{ok} flag to check if sampler is built
\section{run\_smc\_sampler}

 \begin{lstlisting}
   ok = matbiips('run_smc_sampler', id, nb_part, seed, ess_threshold, resample_type)
 \end{lstlisting}

   \begin{description}
     \item[\texttt{id}]: the number of the console
     \item[\texttt{nb\_part}]: number of particles
     \item[\texttt{seed}]: seed for random number generator
     \item[\texttt{ess\_threshold}]: threshold value for the Effective Sample Size
     \item[\texttt{resample\_type}]: string belonging to \texttt{ 'stratified', 'systematic', 'residual', 'multinomial' }

   \end{description}

  \texttt{RUN\_SMC\_SAMPLER} run the SMC with \texttt{nb\_part} particles. One can select the resampling method using \texttt{resample\_type}
  \section{get\_log\_norm\_const}

 \begin{lstlisting}
  log_norm = matbiips('get_log_norm_const', id)
 \end{lstlisting}
 \texttt{GET\_LOG\_NORM\_CONST} returns the log normalizing constant relative to the console \texttt{id}

 \section{get\_filter\_monitors}

 \begin{lstlisting}
  filter_monitors = matbiips('get_filter_monitor', id)
 \end{lstlisting}

 \texttt{GET\_FILTER\_MONITORS} dumps the filters monitors into the \texttt{filter\_monitors} double nested struct: for each monitored variable , we have
  a nested structure containing the following fields:
  \begin{itemize}
   \item \texttt{values}: array of vector of monitored values (size: \texttt{[dim, n\_part]} where \texttt{dim} is the dimension of the node array)
   \item \texttt{weigths}: array of weights (same dimensions as values)
   \item \texttt{ess}: array of effective  sample size (size: \texttt{[dim, n\_part]})
   \item \texttt{discrete}: array of booleans (size: \texttt{[dim]}). 1 if the variables is discrete, 0 otherwise
   \item \texttt{conditionals}: cell array (size: \texttt{[dim]}) of conditioning variables (observations). Each element contains a cells of strings with the respective list of conditioning variables of  the monitored node array element.
   \item \texttt{name}: name of the variable
   \item \texttt{lower}: vector of lower bounds
   \item \texttt{upper}: vector of upper bounds
   \item \texttt{type}: string 'filtering'
   \end{itemize}

  For instance, if one wants to get all monitored values for variable \texttt{'x[1:4]'}, he could use
 \begin{lstlisting}
>> filter_monitors.('x[1:4]')

ans =

      values: [4x100 double]
     weights: [4x100 double]
         ess: [4x1 double]
    discrete: [4x1 double]
        name: 'x'
       lower: 1
       upper: 4
        type: 'filtering'

 \end{lstlisting}

 \section{get\_gen\_tree\_smooth\_monitors}

 \begin{lstlisting}
  smooth_monitors = matbiips('get_gen_tree_smooth_monitors', id )
 \end{lstlisting}

 \texttt{GET\_GEN\_TREE\_SMOOTH\_MONITORS} dumps the smooth monitors into the \texttt{smooth\_monitors} double nested struct: for each monitored variable , we have
  a nested structure containing the following fields:

  \begin{itemize}
   \item \texttt{values}: array of vector of monitored values (size: \texttt{[dim, n\_part]} where \texttt{dim} is the dimension of the node array)
   \item \texttt{weigths}: array of weights (same dimensions as values)
   \item \texttt{ess}: array of effective  sample size (size: \texttt{[dim, n\_part]})
   \item \texttt{discrete}: array of booleans (size: \texttt{[dim]}). 1 if the variables is discrete, 0 otherwise
   \item \texttt{conditionals}: cell of strings with the list of conditioning variables (observations) of the monitored variable.
   \item \texttt{name}: name of the variable
   \item \texttt{lower}: vector of lower bounds
   \item \texttt{upper}: vector of upper bounds
   \item \texttt{type}: string 'smoothing'
   \end{itemize}

  For instance, if one wants to get all monitored values for variable \texttt{'x[1:4]'}, he could use
 \begin{lstlisting}
>> smooth_monitors.('x[1:4]')

ans =

      values: [4x100 double]
     weights: [4x100 double]
         ess: [4x1 double]
    discrete: [4x1 double]
        name: 'x'
       lower: 1
       upper: 4
        type: 'smoothing'
 \end{lstlisting}

 \section{get\_backward\_smooth\_monitors}

 \begin{lstlisting}
  backward_smooth_monitors = matbiips('get_backward_smooth_monitors', id )
 \end{lstlisting}

 \texttt{GET\_BACKWARD\_SMOOTH\_MONITORS} dumps the backward smooth monitors into the \texttt{backward\_smooth\_monitors} double nested struct: for each monitored variable , we have
  a nested structure containing the following fields:

  \begin{itemize}
   \item \texttt{values}: array of vector of monitored values (size: \texttt{[dim, n\_part]} where \texttt{dim} is the dimension of the node array)
   \item \texttt{weigths}: array of weights (same dimensions as values)
   \item \texttt{ess}: array of effective  sample size (size: \texttt{[dim, n\_part]})
   \item \texttt{discrete}: array of booleans (size: \texttt{[dim]}). 1 if the variables is discrete, 0 otherwise
   \item \texttt{conditionals}: cell of strings with the list of conditioning variables (observations) of the monitored variable.
   \item \texttt{name}: name of the variable
   \item \texttt{lower}: vector of lower bounds
   \item \texttt{upper}: vector of upper bounds
   \item \texttt{type}: string 'backward\_smoothing'
   \end{itemize}

  For instance, if one wants to get all monitored values for variable \texttt{'x[1:4]'}, he could use
 \begin{lstlisting}
>> backward_smooth_monitors.('x[1:4]')

ans =

      values: [4x100 double]
     weights: [4x100 double]
         ess: [4x1 double]
    discrete: [4x1 double]
        name: 'x'
       lower: 1
       upper: 4
        type: 'backward_smoothing'

 \end{lstlisting}

\section{clear\_filter\_monitors}

 \begin{lstlisting}
  matbiips('clear_filter_monitors', id , release)
 \end{lstlisting}

  \texttt{CLEAR\_FILTER\_MONITORS} removes all filter monitor relative to the console \texttt{id};
  \texttt{release} is a  boolean value indicating to keep the information of which nodes are monitored (if false, in addition to deallocate memory,
  we forgive which nodes are monitored)

\section{clear\_gen\_tree\_smooth\_monitors}

 \begin{lstlisting}
  matbiips('clear_gen_tree_smooth_monitors', id , release)
 \end{lstlisting}

  \texttt{CLEAR\_GEN\_TREE\_MONITORS} removes all smooth monitors relative to the console \texttt{id};
  \texttt{release} is a  boolean value indicating to keep the information of which nodes are monitored (if false, in addition to deallocate memory,
  we forgive which nodes are monitored)

\section{clear\_backward\_smooth\_monitors}

 \begin{lstlisting}
  matbiips('clear_backward_smooth_monitors', id , release)
 \end{lstlisting}

  \texttt{CLEAR\_BACKWARD\_SMOOTH\_MONITORS} removes all bac backward smooth monitors relative to the console \texttt{id};
  \texttt{release} is a boolean value indicating to keep the information of which nodes are monitored (if false, in addition to deallocate memory,
  we forgive which nodes are monitored)

\section{run\_backward\_smoother}


 \begin{lstlisting}
  ok = matbiips('run_backward_smoother', id)
 \end{lstlisting}

  \texttt{RUN\_BACKWARD\_SMOOTHER} run the backward smoother algorithm relative to the console \texttt{id}


\section{get\_sorted\_nodes}


 \begin{lstlisting}
  sort_nodes= matbiips('get_sorted_nodes, id)
 \end{lstlisting}
  \texttt{GET\_SORTED\_NODES} retrieves a struct \texttt{sort\_nodes} containing the following fields
  \begin{itemize}
   \item \texttt{id}: vector of number for nodes
   \item \texttt{names}: cell of strings containing the names of nodes
   \item \texttt{type}: cell of strings containing the type of nodes \texttt{'Stochastic', 'Constant', 'Logical'}
   \item \texttt{observed}: vector of boolean flags to indicate if the node is observed or not
   \end{itemize}

\section{get\_nodes\_samplers}

 \begin{lstlisting}
  nodes_samplers = matbiips('get_nodes_samplers, id)
 \end{lstlisting}

  \texttt{GET\_NODES\_SAMPLERS} retrieves a struct \texttt{nodes\_samplers} containing the following fields:

  \begin{itemize}
   \item \texttt{iteration}: vector of integers contaning the numbers of iterations for each node
   \item \texttt{sampler}: cell of strings containing the name of samplers (e.g \texttt{'Conjugate Normal (with known precision)'})
   \end{itemize}


\section{print\_graphviz}

 \begin{lstlisting}
   matbiips('print_graphviz', id, filename)
 \end{lstlisting}

  \texttt{PRINT\_GRAPHVIZ} save a graphical represention of the model of console \texttt{id} in the dot file \texttt{filename}

\section{change\_data}

 \begin{lstlisting}
 change_ok = matbiips('change_data', id, var, lower, upper, values, mcmc)
 \end{lstlisting}

 \texttt{CHANGE\_DATA} changes the values of some variables in the model: one could use the following
  arguments

  \begin{itemize}
   \item \texttt{id}: console id
   \item \texttt{var}: string name of the variable to change
   \item \texttt{lower}: vector of lower bounds for the variable to change
   \item \texttt{upper}: vector of upper bounds for the variable to change
   \item \texttt{values}: vector or matrix of new values
   \item \texttt{mcmc}: boolean to enforce some checks about discretness of variables if using Biips embedded in a mcmc algorithm
   \end{itemize}

\section{sample\_data}

 \begin{lstlisting}
 sampled_values = matbiips('sample_data', id, var, lower, upper, seed)
 \end{lstlisting}

  \texttt{SAMPLE\_VALUES} samples values for some variables in the model \texttt{id}

  \begin{itemize}
   \item \texttt{id}: console id
   \item \texttt{var}: string name of the variable to sample
   \item \texttt{lower}: vector of lower bounds for the variable to sample
   \item \texttt{upper}: vector of upper bounds for the variable to sample
   \item \texttt{seed}: seed for random number generation
   \item \texttt{samples\_values} : vector or matrices of sampled values
   \end{itemize}

   \section{get\_log\_prior\_density}

 \begin{lstlisting}
  log_prior_density = matbiips('get_log_prior_density', id, var, lower, upper)
 \end{lstlisting}
    \texttt{GET\_LOG\_PRIOR\_DENSITY} retrieves the log prior density of the model \texttt{id}

 \begin{itemize}
   \item \texttt{id}: console id
   \item \texttt{var}: string name of the variable to sample
   \item \texttt{lower}: vector of lower bounds for the variable to sample
   \item \texttt{upper}: vector of upper bounds for the variable to sample
   \item \texttt{log\_prior\_density} : scalar value of the log prior density
   \end{itemize}

\section{is\_smc\_sampler\_at\_end}

 \begin{lstlisting}
  ok = matbiips('is_smc_sampler_at_end', id)
 \end{lstlisting}
 \texttt{IS\_SMC\_SAMPLER\_AT\_END} checks that the smc sampler algorithm associated to the console \texttt{id} has terminated.


\section{message}

 \begin{lstlisting}
  matbiips('message', id, msg)
 \end{lstlisting}

  \texttt{MESSAGE} writes the string \texttt{msg} in the console \texttt{id}

\section{set\_log\_norm\_const}

 \begin{lstlisting}
  matbiips('set_log_norm_const', id, log_norm)
 \end{lstlisting}

  \texttt{SET\_LOG\_NORM\_CONST} sets the value of the log normalisation constant

\section{make\_progress\_bar}

 \begin{lstlisting}
  bar_id = matbiips('make_progress_bar', num, progress_string, name)
 \end{lstlisting}

 \texttt{MAKE\_PROGRESS\_BAR} builds a progress bar in the current output stream. arguments are the following
 \begin{itemize}
  \item \texttt{num} : the number max of iterations (ie 100\% value)
  \item \texttt{progress\_string} : one character string used for progression (e.g '*', '+' or 'x')
  \item \texttt{name} : string for the name of bar
  \item \texttt{bar\_id} : number of the bar, in the table of bars
  \end{itemize}

\section{clear\_progress\_bar}

 \begin{lstlisting}
  matbiips('clear_progress_bar', bar_id)
 \end{lstlisting}
  \texttt{CLEAR\_PROGRESS\_BAR} clears the bar \texttt{bar\_id}

\section{advance\_progress\_bar}

 \begin{lstlisting}
  matbiips('advance_progress_bar', bar_id, count)
 \end{lstlisting}

 \texttt{ADVANCE\_PROGRESS\_BAR} make a progress of \texttt{count} units for the bar \texttt{bar\_id}


\section{is\_gen\_tree\_smooth\_monitored}

 \begin{lstlisting}
  ok = matbiips('is_gen_tree_smooth_monitored', id, names, lower, upper, released)
 \end{lstlisting}

\texttt{IS\_GEN\_TREE\_SMOOTH\_MONITORED} checks that smooth monitoring of some variable is finished: arguments are the following

 \begin{itemize}
  \item \texttt{id} : id of the console
  \item \texttt{names} : cell of strings indicating variables to check
  \item \texttt{lower} : cell of integers indicating lower bounds in variable monitoring
  \item \texttt{upper} : cell of integers indicating upper bounds in variable monitoring
  \item \texttt{released} : boolean value indicating to keep the information of which nodes are monitored (if false, in addition to deallocate memory,
  we forgive which nodes are monitored)
 \end{itemize}


\section{sample\_gen\_tree\_smooth\_particle}

 \begin{lstlisting}
  matbiips('sample_gen_tree_smooth_particle', id, seed)
 \end{lstlisting}

 \texttt{SAMPLE\_GEN\_TREE\_SMOOTH\_PARTICLE} samples the values of the smooth monitors using \texttt{seed} for random number generation

\section{get\_sampled\_gen\_tree\_smooth\_particle}
 \begin{lstlisting}
  sampled_value = matbiips('get_sampled_gen_tree_smooth_particle', id)
 \end{lstlisting}

 \texttt{GET\_SAMPLED\_GEN\_TREE\_SMOOTH\_PARTICLE} retrieves the sample values of smooth monitor in a structure which has for fields the names
 of monitored variables.

\section{set\_sampled\_gen\_tree\_smooth\_particle}

 \begin{lstlisting}
  sampled_value = matbiips('set_sampled_gen_tree_smooth_particle', id, values)
 \end{lstlisting}
  \texttt{SET\_SAMPLED\_GEN\_TREE\_SMOOTH\_PARTICLE}
\section{weighted\_quantiles}

 \begin{lstlisting}
  quantiles = matbiips('weighted_quantiles', values, weights, probs)
 \end{lstlisting}

 \texttt{WEIGHTED\_QUANTILES} computes the weigthed quantiles using Boost accumulator


\section{cell2struct\_weak\_names}

 \begin{lstlisting}
  weak_cell = matbiips('weighted_quantiles', values, names)
 \end{lstlisting}

  \texttt{CELL2STRUCT\_WEAK\_NAMES} does the same jobs as \texttt{cell2struct} - build a struct from 2 cell arrays - but names of fields could be
  illegals, i.e.  names containing \texttt{[,:,]}

\section{add\_function}

 \begin{lstlisting}
  matbiips('add_function', func, nb_args, fun_dim, fun_eval, fun_check, fun_discrete)
 \end{lstlisting}

 \texttt{ADD\_FUNCTION} adds a matlab function in the functions table. After that, the user could use the function in BUGS model. The arguments are the following

 \begin{itemize}
  \item \texttt{name\_func} : name of the function in bug models
  \item \texttt{nb\_args} : number of arguments of the function
  \item \texttt{fun\_dim} : name of the M-file function checking the size of arguments and returning a vector of dimensions of the output
  \item \texttt{fun\_eval} : name of the M-file function which computes the value of the function
  \item \texttt{fun\_check} : name of the M-file function checking the arguments
  \item \texttt{fun\_discrete} : name of the M-file function which returns a boolean indicating the discretness of the output parameter of \texttt{add\_function}
\end{itemize}

\end{document}

\documentclass[12pt,twoside,french]{article}
\usepackage[a4paper]{geometry}
\usepackage[latin1]{inputenc} % ou \usepackage[utf8]{inputenc}
\usepackage[T1]{fontenc} % ou \usepackage[OT1]{fontenc}

\usepackage{hyperref}
\usepackage[frenchb]{babel} % optionnel
%\usepackage[round]{natbib} % optionnel
\usepackage{amsmath}
\usepackage{amsfonts}
\usepackage{amssymb}
\usepackage{graphicx}
\usepackage{algorithm}
\usepackage{fancyhdr}
\usepackage{subfigure}
\usepackage{epstopdf}
\usepackage{fancybox}
\usepackage{verbatim}
%\usepackage{float}
%\usepackage{here}
\usepackage{color}
%%%%%%%%%%%%%%%%%%%
%%%%% en tete marc
%%%%%%%%%%%%%%%%%%
\usepackage{listings}
\usepackage{relsize}
\definecolor{dkgreen}{rgb}{0,0.6,0}
\definecolor{gray}{rgb}{0.8,0.8,0.8}
\definecolor{mauve}{rgb}{0.58,0,0.82}
\newcommand{\biips}{{\bf Biips\ }{}}
\newcommand{\matlab}{{\bf MATLAB\ }{}}
\lstset{ %
  framerule=0pt,
  basicstyle=\relsize{-2}\ttfamily,  % the size of the fonts that are used for the code
  backgroundcolor=\color{gray},  % choose the background color. You must add \usepackage{color}
  showspaces=false,               % show spaces adding particular underscores
  showstringspaces=false,         % underline spaces within strings
  showtabs=false,                 % show tabs within strings adding particular underscores
  %frame=single,                   % adds a frame around the code
  rulecolor=\color{black},        % if not set, the frame-color may be changed on line-breaks within not-black text (e.g. commens (green here))
  breakatwhitespace=false,        % sets if automatic breaks should only happen at whitespace
  keywordstyle=\color{blue},      % keyword style
  commentstyle=\color{dkgreen},   % comment style
  stringstyle=\color{mauve}
}
\pagestyle{empty}
\begin{document}

\begin{center}
{\Large matBiips tutorial}
\end{center}


\section{Matlab Interface for \biips{}}

  Biips has a \matlab interface, based on mexfiles (\matlab Executables). Mexfiles are binaries dynamically loaded by the \matlab interpretor, and
  enable us to call speed C or Fortran code. Since Biips, is statically included in the mexfiles, the only dependencies we need at execution, 
  are those from \matlab (libmat, libmex and libstdc++)
  

  \subsection{Application Program Interface}

   The \matlab Biips API, contains about ten functions permitting to the user to build a graphical bayesian model, run smc algorithms
   and compute statistics.

  \begin{figure}[H]
  \begin{tabular}{|l|l|}
  \hline
   nom de fonction & action(s) r�alis�e(s) \\
  \hline
   \texttt{biips\_init} & initialisation \\
   \texttt{biips\_model} & model compilation \\
   \texttt{biips\_smc\_samples} & run SMC algorithm following the model \\
   \texttt{biips\_diagnostic} & results diagnostic \\
   \texttt{biips\_summary} & compute a statistical summary \\ 
   \texttt{biips\_density} & univariate density estimator computation \\ 
   \texttt{biips\_clear} & clear the compiled model \\
  \hline
   \texttt{biips\_add\_function} & add an external function \\
   \texttt{biips\_build\_sampler} & assign samples to the graph nodes \\ 
   \texttt{biips\_get\_nodes} & retrieve information about nodes \\ 
   \texttt{biips\_get\_data} & retrieve model data (constants and observations) \\
   \texttt{biips\_get\_variables} & retrieve model variables name \\ 
   \texttt{biips\_print\_dot} & write the model graph in a \texttt{dot} file\\
   \hline
  \end{tabular}
  \end{figure}

   First category functions are well suited for basic usage, whereas thoses from second are reserved for a advanced usage
   
   \subsubsection{Example}

   Let us assume that the user of \biips{} wrote his model in \verb!model.bug! file , one example of a matbiips session
   could be the following 

\begin{lstlisting}[language=matlab]
% initial loading 
addpath('repertoire/de/matbiips');
biips_init;
% data definition 
data = struct('var1', value, 'var2', value, ...);
% model compilation
[p, data] = biips_model('model.bug', data);
% SMC run
nb_part = 1000;
out_smc = biips_smc_samples(p, {'var3', 'var4[1,1:10]', ...}, nb_part, 'fsb')
% statistics computation
out_summ = biips_summary(out_smc, {}, 'fsb');
\end{lstlisting}

   Concerning functions previously used, we can isolate two rules :
     \begin{itemize}
       \item first output argument \texttt{p} returned by \texttt{biips\_model} is the index of the console in the console table 
       of \biips{}; In the following, it will be used by the other functions of the API to exchange messages with \biips{}
       , so you {\bf have to save it after a \texttt{biips\_model} call.}
       \item string arguments like \verb!'fs'! or \verb!'fsb'! match with the kind of the SMC algorithm : 
        {\em filtering} (\verb!f!), {\em smoothing} (\verb!s!) or {\em backward smoothing} (\verb!b!) 
     \end{itemize}


  \subsection{About output structures}
  
  The function \verb!biips_smc_samples! enable us to monitor some variables, by listing
  them in a \verb!cell! of strings given as a second argument. This function and others, using the same mechanism, return a structure with fields
  which are the name of monitored variables. Each field, contains subfields for each variant ('fsb') of the SMC (given by the $4^{\mbox{th}}$
  argument) 
  \paragraph{}
  
  the user can also monitor a subpart of a variable : e.g., if \verb!X! is an array $5\times100$, it is possible to monitor the $5^{\mbox{th}}$ component 
  in the following manner 
\begin{lstlisting}[language=matlab]
out_smc = biips_smc_samples(p, {'X[5,1:100]'}, nb_part, 'fsb')
\end{lstlisting}
  
  \subsection{"weak" structures names} 
     
     The \matlab interface was created after the R interface. R has slacker rules for name fields than \matlab, you can use
     brackets or commas in the fields names of a cell. In \matlab, if you wan to use these illegal names, we provide 
     an {\em ad-hoc} function
\begin{lstlisting}[language=matlab]
biips_cell2struct({ fields }, { weak_names }) 
\end{lstlisting}
     which has the same signature that \verb!cell2struct!.
     
  \paragraph{}
     After one can retrieve filter results (variante\verb!f!) of \verb!'X[5,1:100]'! with the following call : 
\begin{lstlisting}
>> out_smc.('X[5,1:191]').f

ans = 

      values: [1x100x1000 double]
     weights: [1x100x1000 double]
         ess: [1x100 double]
    discrete: [1x100 double]
        name: 'X'
       lower: [5 1]
       upper: [5 100]
        type: 'filtering'
\end{lstlisting}

   \subsection{Internal functions}

    The mexfile used to call \biips is \texttt{inter\_biips.cpp} : in that mexfile, we have one entry point function \texttt{inter\_biips} which takes as a first 
    argument the name of the \biips routine we want to call and after the arguments to pass to the routine. Here there is a list of theses internal routines

  \begin{figure}[H]
   \tiny
  \begin{tabular}{|l|l|l|}
    name & args & return value \\
   \hline
    make\_console & & console\_id 
   \hline
    clear\_console & console\_id  &  \\
   \hline
    clear\_all\_console & &  \\ 
   \hline
    check\_model & console\_id, file\_name & \\
   \hline
    compile\_model & console\_id, var\_values\_struct, sample\_flag, seed & \\
   \hline
    get\_data & & \\
   \hline
    load\_module & \\
   \hline
   verbosity & \\
   \hline
    get\_variable\_names & \\
   \hline
    set\_default\_monitors & \\
   \hline
    set\_filter\_monitors & \\
   \hline
    set\_gen\_tree\_smooth\_monitors & \\
   \hline
    set\_backward\_smooth\_monitors & \\
   \hline
    build\_smc\_sampler & \\
   \hline
    is\_sampler\_built & \\
   \hline
    run\_smc\_sampler & \\
   \hline
    get\_log\_norm\_const & \\
   \hline
    get\_filter\_monitors & \\
   \hline
    get\_gen\_tree\_smooth\_monitors & \\
   \hline
    get\_backward\_smooth\_monitors & \\
   \hline
    clear\_filter\_monitors & \\
   \hline
    clear\_gen\_tree\_smooth\_monitors & \\
   \hline
    clear\_backward\_smooth\_monitors & \\
   \hline
    run\_backward\_smoother & \\
   \hline
    get\_sorted\_nodes & \\
   \hline
    get\_nodes\_samplers & \\
   \hline
    print\_graphviz & \\
   \hline
    change\_data & \\
   \hline
    sample\_data & \\
   \hline
    get\_log\_prior\_density & \\
   \hline
    is\_smc\_sampler\_at\_end & \\
   \hline
   message & \\
   \hline
    set\_log\_norm\_const & \\
   \hline
    make\_progress\_bar & \\
   \hline
    clear\_progress\_bar & \\
   \hline
    advance\_progress\_bar & \\
   \hline
    is\_gen\_tree\_smooth\_monitored & \\
   \hline
    get\_sampled\_gen\_tree\_smooth\_particle & \\
   \hline
    sample\_gen\_tree\_smooth\_particle & \\
   \hline
    set\_sampled\_gen\_tree\_smooth\_particle & \\
   \hline
    weighted\_quantiles & \\
   \hline
    cell2struct\_weak\_names & \\
   \hline
    add\_function & \\
   \hline
  \end{tabular}
  \end{figure}


\end{document}
